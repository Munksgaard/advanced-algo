%Exercise 3.5 content

\subsection{}

We want to translate the different properties of the rectilinear graph
$G = (V, E)$ into a graph $G'$ representing a minimum-cost-flow
problem. For simplicity we'll allow bidirectional edges; they can
easily be eliminated by introducing an intermediate vertex, so this
shouldn't be a problem.

To do this, each face in $G$ is translated into a vertex in the
minimum-cost-flow problem with demand $-4$ if the face is external and
$4$ otherwise. Vertices in $G'$ representing faces that share edges in
$G$ are connected bidirectionally to each other with edges that have $cost = 1,
capacity = \infty $.

Vertices in $G$ are also added to $G'$ as vertices with demand $0$ if
they have degree $2$, demand $2$ if they have degree $3$, and demand
$4$ if they have degree $4$. Now, the newly constructed vertices in
$G'$ representing vertices in $G$, are connected bidirectionally to
the vertices representing faces in $G$ for which they appear in the
boundary cycle. These edges are assigned capacities $1$ and cost $0$.
