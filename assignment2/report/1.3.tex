\subsection{ILP lower bound}

We first need to define a function, given a vertex, that returns all vertices
in the vicinity of the argument vertex, where the vicinity distance is
\texttt{d}:

\begin{align}
    v(i) = \{j \in J | d_{ij} \leq d\}
\end{align}
We then define and express the TCP instance as a integer linear programming
problem instance.

\begin{align*}
    \min             & \sum_{i,j, i \not= j} x_{ij} c_{ij} & \\
    \text{subject to}& &\\
                     & x_{ij} \in \setN{0,1}, &\hspace{2em} i,j \in V\\
                     & \sum_{j=0,i\neq j}^{n} x_{ij} + \sum_{j \in v(i)} \sum_{k = 1, i \neq k}^n x_{kj}          &\geq 1, \hspace{2em} i = 1,\hdots,n \\
                     & \sum_{j = 0}^n x_{ij} + x_{ji}                   &\leq 2, \hspace{2em} i = 1,\hdots,n \\
                     & \sum_{j = 0}^n x_{ij} - x_{ji}                   &= 0, \hspace{2em} i = 1,\hdots,n \\
                     & \sum_{i \in S} \sum_{j \in V \setminus S} x_{ij} &\geq 2, \hspace{2em} \forall S \subset V,~ V = \setN{i,j | x_{ij} = 1}
\end{align*}
The first constraint makes sure that vertices, in the vicinity of vertex
\texttt{i}, can be seen from a vertex on the solution path.

The second and third constraint guarantee that the degree of all combinations
of vertices is below two - so we can never have two circles intersecting or
otherwise have edges with more than on ingoing and one outgoing edge. The
third constraint specifically specifies that either the vertices have one in-
and outgoing edge, or they have none.

The fourth constraint takes care of potential sub tours in the solution
circle. We basically make a cut of all the vertices that contains in-and
outgoing edges, and then check that for any possible cut that the number of
edges intersecting the cut, is exactly equal to two. That way, we make sure
the integer linear programming problem solver doesn't end up with multiple
circles.
