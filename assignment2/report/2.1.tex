\subsection{Implementation}
We have chosen to remove the no-subtour constraint. This is possible since
we are implementing a lower bound by using a minimization ILP. Removing the
constraint of not having subtours simply adds extra elements to the set of
feasible solutions. If some subtour has a higher cost, it will not affect
the outcome of the ILP since we are minimizing and hence ``skipping'' that
particular subtour. There could however be solutions with subtours having a
lower cost, but this would only lower the lower bound and hence not falsely
eliminate branches with possible optimal solutions. The removal of the subtour
constraint will likely lead to a higher amount of branching, but we gain a
simpler implementation and a simpler ILP to solve.

Likewise we choose to relax the problem from having integer values of our
representation (in fact binary values) to having real valued variables between
0 and 1. This relaxation will inevitably give us less tight bounds than
for integer problems but the problem will be easier to solve since integer
problems are much harder to solve than real valued problems in general.

We have implementing the lower bound fully, which consists of our first three
constraints; namely the vicinity and the two vertex degree constraints. To
do this, we have made a \texttt{getIndex} method, that takes in two integers
corresponding to the two vertices and returns the mapped index to a row, which
does not contain edges to itself, so $x_{ii}$ is not contained in the row.

Our vicinity constraint has been implemented by iterating through all
vertices, from which we check if each of the nodes' vicinity $v(i)$ is visited
- this can be done, by checking if $\sum_{j \in v(i)} \sum_{k=0}^n x_{jk} +
x_{kj} >= 2$.

For the two degree constraints, we simply iterate over all vertices and
then over all vertices again, such that $i \ne j$, we then simply get the
indices for the edge $\{i,j\}$ and $\{j,i\}$ and set them each to $1$ for the
second constraint. For the third constraint, we do the same, only setting the
$\{j,i\}$ edge to $-1$, to get the minus relation.\\

