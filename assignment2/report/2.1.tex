\subsection{Implementation}
We have chosen to remove the no-subtour constraint. This is possible since we
are implementing a lower bound by using a minimization ILP. Removing the
constraint of not having subtours simply add extra elements to the set of
feasible solutions. If some subtour has a higher cost, it will not affect
the outcome of the ILP since we are minimizing and hence ``skipping'' that
particular subtour. There could however be solutions with subtours having
a lower cost, but this would only lower the lower bound and hence not
falsely eliminating branches with possible optimal solutions. The removal
of the subtour constraint will likely lead to a higher amount of
branching, but we gain a simpler implementation and a simpler ILP to
solve.

Likewise we choose to relax the problem from having integer values of our
representation (in fact binary values) to having real valued variables
between 0 and 1. This relaxation will inevitably give us less tight bounds
than for integer problems but the problem will be easier to solve since
integer problems are much harder to solve than real valued problems in
general.

We have not succeeded in implementing the lower bound fully. This was
caused by an early design decision of only having each edge represented once,
which we found out late in the developing process was a mistake since we
had no way of guaranteeing that each vertex had exactly degree 0 or 2.

Instead we should have had each edge represented twice and hence given two
vertices $i$ and $j$, $i \not= j$ we would have $x_{ij}$ and $x_{ji}$,
which makes it possible to model the constraint described above.

Using this new design decision we could also implement the constraints
directly from the equations above with a bit of fiddling with the exact
indexing implementation.

In case we get a re-submission of the assignment we would like to have
pointers/feedback on our intended implementation instead of the one
supplied in the code.

Note: We have made sure the code is actually compiling and is executeable.
The executing requires lp-solve 5.5 to be installed. To execute the code
simply unzip the attached archive and execute \texttt{make} in a terminal.
Otherwise simply compile and run the java program with the classpath
attribute \texttt{-cp lpsolve55j.jar:.}.
