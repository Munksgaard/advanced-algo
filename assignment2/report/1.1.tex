\subsection{}

Let $G = (V,E)$ be a complete undirected weigthed graph, $c$ be the
cost function $V \times V \to \mathbb{Z}$, and a non-negative integer
$d$. Find the shortest cycle $\sigma$ in $G$, such that for all $v \in
V$, $v \in \sigma$ or there exists a vertex $v' \in \sigma$ such that
$c_{v'v} \leq k$.

The decision problem can be formulated as such: Given a complete
undirected weighted graph $G=(V,E)$ with weight function $c ~ : ~ V
\times V \to \mathbb{Z}$, and non-negative integers $d,k$, does there
exist a cycle $\sigma$ in $G$ with total weight $w$ such that $w \leq
k$, and that for all $v \in V$, $v \in \sigma$ or there exists a
vertex $v' \in \sigma$ such that $c_{v'v} \leq k$.

\begin{align*}
  TCP = \{ \langle G, c, d, k \rangle : ~ & G = (V,E) ~ \text{is a complete
    undirected graph}, \\
  & c ~ \text{is a function from} ~ V \times V \to \mathbb{Z}, \\
  & d \in \mathbb{Z}, \\ 
  & k \in \mathbb{Z} \textrm{, and} \\
  & G ~\text{has a traveling-couples tour with cost at most} ~ k\} .\\
\end{align*}

To show that TCP is in the class of NP-complete problems, we first show that
TCP is in NP. We do this by constructing an algorithm that verifies whether a
given certificate decides the problem:\\

Given an instance of the TCP problem and an ordered certificate T, we check
that the following three cases holds:

\begin{enumerate}
  \item {For all $v \in V\backslash T$: Check that $\exists u \in T$, such
  that $c(v,u) \leq d$.}
  \item {$\sum_{\{u,v\} \in T} c(u,v) \leq k$.}
  \item {T contains only one duplicate, namely $T(1) = T(last)$.}
\end{enumerate}

Clearly all of the above cases can be solved in polynomial time. If just
one of these cases are unsatisfied, the algorithm outputs \texttt{False},
otherwise \texttt{True}. So we have shown that TCP lies in the class of NP
problems.\\

We will now show that $\text{HAM-CYCLE} \leq_p \text{TCP}$.

Let $G = (V,E)$ be an instance of HAM-CYCLE. We construct an instance
of TCP as follows. We form the complete graph $G' = (V,E')$, where $E'
= \{(i,j)~:~i,j\in V ~ \text{and}~ i \neq j \}$, and we define the
cost function $c$ by

\begin{align*}
  c(i,j) = \begin{cases} & 0 ~ \text{if} ~ (i, j) \in E~ , \\
    & 1 ~ \text{if} (i, j) \notin E~.\end{cases}\\
\end{align*}

The instance of TCP is then $\langle G', c, 0, 0 \rangle$., which we
can easily create in polynomial time.

We now show that graph $G$ has a hamiltonian cycle if and only if
graph $G'$ has a tour of cost at most $0$. Suppose that graph $G$ has
a hamiltonian cycle $h$. Each edge in $h$ belongs to $E$ and thus has
cost $0$ in $G'$. Since we've set $d=0$, we also have that two
vertices $v_1, v_2$ where $v_1 \neq v_2$ never satifies $c_{v_1v_2} \leq d$,
which means that all vertices in $h$ are also in $V$. Thus, $h$ is a
tour in $G'$ with cost $0$. Conversely, suppose that graph $G'$ has a
tour $h'$ of cost at most $0$. Since the cost of the edges in $E'$ are
$0$ and $1$, the cost of tour $h'$ is exactly $0$ and each edge on the
tour must have cost 0. And since the cost of the edges in $E$ are $0$,
we can completely ignore $d$. Therefore, $h'$ contains only edges in
$E$.

Thus the time complexity of $TCP$ is $O(k^n)$ for some constant $k$.
