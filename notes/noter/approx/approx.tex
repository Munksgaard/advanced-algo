\documentclass[a4paper]{article}

%Packages
\usepackage[english]{babel}
\usepackage[ansinew]{inputenc}
\usepackage{amsmath, amssymb, amsthm}
\usepackage{cancel}
\usepackage[T1]{fontenc}
\usepackage{graphicx}
\usepackage[stable]{footmisc}
\usepackage{lastpage}
\usepackage{ulem}
\usepackage{listings}
\usepackage[usenames,dvipsnames]{color}
\usepackage{clrscode3e}
\usepackage{multirow}
\usepackage{hyperref}

%Listings
\definecolor{listingShade}{RGB}{245,245,245}
\lstset{ %
%language=PHP, %LANGUAGE HERE! :D
basicstyle=\ttfamily,
numbers=left,
numberstyle=\ttfamily\footnotesize,
stepnumber=1,
numbersep=5pt,
backgroundcolor=\color{listingShade},
showspaces=false,
showstringspaces=false,
showtabs=false,
frame=tb,
tabsize=8,
captionpos=b,
breaklines=true,
breakatwhitespace=false,
title=\lstname
}

%Theorems
\theoremstyle{definition}
\newtheorem{thm}{Theorem}[section]
\theoremstyle{definition}
\newtheorem{lem}{Lemma}[section]
%\theoremstyle{definition}
%\newtheorem{defi}{Definition}%[section]

%Math sets
\newcommand{\N}{\mathbb N} %the natural numbers
\newcommand{\Z}{\mathbb Z} %the integers
\newcommand{\Q}{\mathbb Q} %the rational numbers
\newcommand{\R}{\mathbb R} %the real numbers
\newcommand{\C}{\mathbb C} %the complex numbers

%Other commands
\newcommand{\ul}{\underline}
\newcommand{\ol}{\overline}
\newcommand{\mono}[1]{{\ttfamily#1}}

%Title stuff
\title{Advanced algorithms\\Approximation algorithms Notes}
\author{S�ren Dahlgaard}

\begin{document}

\maketitle

\section{Disposition}
\begin{enumerate}
\item Motivation. NP-Hard, probably intractable
\item Approximation factor $\max(C/C^*, C^*/C)$.
\item Approximation schemed $(1+\epsilon)$-approximation. Polynomial in
    $\epsilon$.
\item Randomized. Expected value $C \Rightarrow$ $\max(C?C^*,\ldots$
\item Simple example Vertex cover.
\item Say that linear programming relaxation can be okay.
\item Set-covering.
\item Proof of set-covering.
\end{enumerate}


\section{summary}

Just because a problem is NP-Complete and (probably) intractable we don't give
up. if $C^*$ is the value (or cost) of the optimal solution and we have an
algorithm that produces another solution $C$ we say that the algorithm has an
approximation ratio of $\rho(n)$ if
\[\max\left(\frac{C}{C^*}, \frac{C^*}{C}\right)\le\rho(n)\]
We call such an algorithm a $\rho(n)$-approximation algorithm. We assume that
a solution has positive cost.

We also talk about approximation schemes. These take as input an $\epsilon > 0$
such that the scheme is a $(1+\epsilon)-$approximation algorithm.
The running time of such a scheme can grow rapidly as $\epsilon$ gets smaller:
$O(n^{2/\epsilon})$.

We call a scheme fully polynomial if its running time is polynomial in both
$1/\epsilon$ and $n$. eg. $O((1/\epsilon)^2n^3)$.

\section{Vertex cover}
An approximation scheme takes an edge $(u,v)\in E$ arbitrarily and adds
$u,v$ to $S$. It then removes all edges in $E$ incident on either $u$ or $v$
and picks another edge if one is present.

Let $A$ be the edges checked by the algorithm. Any vertex cover must include
either $u$ or $v$, so $|C^*| \ge |A|$. We have that $|C| = 2|A|$, so
$|C|\le 2|C^*|$. Thus this algorithm is a 2-approximation.

\section{TSP}
We focus on TSP with the triangle inequality. That is
$c(u,w) \le c(u,v) + c(v,w)$ for all $u,v,w\in V$. The approximation
algorithm is to first create a MST of $G$. Then visit the vertices of a
preorder walk of this MST.

It is clear that the size of a MST $T$ is less than the size of an optimal
traveling salesman tour. Eg. removing an edge from $C^*$ yields a spanning
tree which certainly cannot have lower weight than $T$.
A preorder of the tree visits every vertex and uses every edge of $T$ exactly
twice. Let this full walk be $W$, we have $c(W) \le 2c(T)\le 2c(C^*)$. $W$ is
however likely not a tour, so we delete duplicate vertices and the triangle
inequality gives us $c(C)\le c(W)\le 2c(C^*)$.

\subsection{General TSP}
We can show that an approximation algorithm for the general TSP if $P\ne NP$.
We could use this approximation algorithm to solve the ham-cycle problem.

Because we know the approximation factor $\rho\ge 1$ we can create a special
graph for which a TSP tour shorter than $\rho|V|$ corresponds to a
hamiltonian cycle. Just use the following cost function for $G'$:

\[
c(u,v) =
\begin{cases}
1 & \text{if } (u,v)\in E \\
\rho|V| + 1 & \text{otherwise}
\end{cases}
\]

Any tour using edges only in $E$ will have length $|V|$. Any tour using at
least one edge not in $E$ will have length at least
$(\rho|V| + 1) + (|V| - 1) > \rho|V|$, so the approximation algorithm cannot
return a tour using edges not in $E$ or it would not be a $\rho$-approximation.

\section{Set-covering}
Given a set of elements $X$ and a family of subsets of $X$ called $F$ we
wish to find the smallest size subset $C\subseteq F$, such that the union of
$C$ is $X$.

Our approximation algorithm works by picking the set that covers most
uncovered elements. It is easy to see that this algorithm runs in polynomial
time.

\begin{proof}
Let $C$ be the subset chosen by the algorithm and let $S_1,\ldots,S_{|C|}$ be
the sets in $C$. We assign a cost of $1$ to each of these sets spread over
the elements that the given set was the first to cover. So for each element
in $x\in S_1$ we have $c_x = 1/|S_1|$. For $x\in S_2$ we have either
$c_x = 1/|S_1|$ if $x$ was first covered by $S_1$. Otherwise we have
$c_x = 1/|S_2\setminus S_1|$, for $S_3$ this will be
$c_x = 1/|S_3\setminus (S_1\cup S_2)|$.

In total we have $|C| = \sum_{x\in X} c_x$.

Each $x$ is in at least one set $S\in C^*$. Therefore we must have

\[\sum_{S\in C^*} \sum_{x\in S} c_x \ge \sum_{x\in X}c_x\]

Which gives:

\[|C| \le \sum_{S\in C^*}\sum_{x\in S} c_x\]

We now use the fact that:
\[\sum_{x\in S}c_x \le H(|S|) = 1 + 1/2 + 1/3 + \ldots + 1/|S|\]

This gives:

\begin{align}
|C| &\le \sum_{S\in C^*} H(|S|) \\
    &\le |C^*| \cdot H(\max\{|S| : S\in F\})
\end{align}

This proofs the theorem.
\end{proof}

\begin{proof}
We also need a proof for the ``fact'' given above. For any set $S\in F$ let
$u_0 = |S|$. Let $u_1 = |S\setminus S_1|$, etc. Pick $k$ such that $u_k = 0$
and $u_{k-1} > 0$. We then have:

\[
\sum_{x\in S} c_x = \sum_{i=1}^k (u_{i-1} - u_i)\cdot
    \frac{1}{|S_i\setminus (S_1\cup S_2 \cup \cdots \cup S_{i-1})}
\]

We also have
\[
|S_i \setminus (S_1\cup \cdots\cup S_{i-1})| \ge
    |S \setminus (S_1\cup \cdots\cup S_{i-1})| = u_{i-1}
\]

Because otherwise $S$ would've been picked before $S_i$. Also note that we
can have some elements in $S_i$ not covered by $S$.

Therefore

\begin{align}
\sum_{x\in S} c_x &\le \sum_{i=1}^k (u_{i-1} - u_i)\cdot 1/u_{i-1} \\
&= \sum_{i=1}^k\sum_{j=u_i + 1}^{u_{i-1}} 1/u_{i-1} \\
&\le \sum_{i=1}^k\sum_{j=u_i + 1}^{u_{i-1}} 1/j \\
&= \sum_{i=1}^k\left(\sum_{j=1}^{u_{i-1}} 1/j - \sum_{j=1}^{u_i} 1/j\right)
\end{align}

This gives us a telescoping series of harmonic numbers that results in
$H(|S|)$.

\end{proof}

This algorithm is therefore a $(\ln |X| + 1)$-approximation. Note that it's
the logarithm of $|X|$ not $|C^*|$!


\section{Randomized approximation}
Maybe skip this or introduce it along with normal approximation algorithms.

If an algorithm yields a result with expected cost $C$ we call it a
randomized $\rho(n)$-approximation algorithm.

If we want to know how many clauses of a 3-CNF we can satisfy at most, we
can try setting each variable to $1$ with probability $1/2$. The probability
that a clause is satisfied is then $1 - (1/2)^3 = 7/8$

Summing these up give $7m/8$ and since $m$ is an upper bound it gives
$m/(7m/8) = 8/7$.

\section{Linear programming}
Let minimum-weight vertex cover be just like vertex-cover except each vertex
has an associated cost $w(v)$, and we want to minimize the cost of the vertex
cover.

We can make the following integer program:
\[\text{min. } \sum_{v\in V}w(v)x(v)\]
\begin{align}
\text{s.t. } x(u) + x(v) &\ge 1 & \text{for } (u,v)\in E \\
x(v) &\in \{0,1\} & \text{for } v\in V
\end{align}

Linear relaxation is when we replace the constraint $x(v)\in \{0,1\}$ with
$0\le x(v) \le 1$.

Our algorithm solves this relaxed program and adds a vertex $v$ to the
solution iff $x(v)\ge 1/2$.

\begin{proof}
Let $C^*$ be the optimal solution to the problem and let $z$ be the optimal
solution to the relaxed linear program. Because $C^*$ is a feasible solution
to this problem we must have $z\le w(C^*)$.

If we by $C$ denote the cover induced by the solution with obj. val. $z$ we
have that $C$ is a vertex cover. Why? Because for any edge $(u,v)\in E$ we
must have either $x(u)\ge 1/2$ or $x(v) \ge 1/2$. Thus every edge is covered.

We have:

\begin{align}
z &=   \sum_{v\in V}w(v)x(v) \\
  &\ge \sum_{v\in V : x(v) \ge 1/2} w(v)\cdot 1/2 \\
  &=   \sum_{v\in C} w(v)\cdot 1/2 \\
  &= 1/2\cdot w(C)
\end{align}

So $w(C) \le 2z \le 2w(C^*)$.

\end{proof}


\section{Subset-sum}
In the optimization version we want to find a subset $S'\subseteq S$ such that
the sum of $S'$ is as large as possible without exceding $t$.

We will give a fully polynomial-time approximation scheme. The idea is for
each $x_i\in S$ we have a list of all possible values by combining
$x_1, \ldots, x_i$. We remove values greater than $t$. This approach takes
exponential time. That is, $L_i = L_{i-1} \cup L_{i-1} + x_i$. If we use a
merge procedure like the one for merge sort this list will remain sorted.

The idea to make this polynomial is to use a parameter $\delta$ such that if
we have two elements in $y\in L_{i-1}$ we might remove it if we have a
$z\in L_i$ such that $y/1+\delta \le z \le y$. For instance if $\delta = 0.1$
we have $11/1.1 \le 10 \le 11$, so $10$ could represent $11$ in our list.

We can trim a sorted list according to a $\delta$ in linear time. If we get an
$0 < \epsilon < 1$ and want a $1+\epsilon$ approximation factor we can use
$\epsilon/2n$ as $\delta$ to trim all lists $L_i$. For instance if
$\epsilon = 0.4$ and $n = 4$ we have $\delta = 0.05 = 0.4/8$.

\begin{proof}
It is obvious that the amount returned by the algorithm is indeed the sum of
some subset because we don't add anything other than subset sums to the lists.

Let $y^*$ be the optimal solution. We know that $z \le y^*$, so we need to
show that $y^*/z \le 1 + \epsilon$ to show correctness of the algorithm.

For $y\in P_i, y \le t$ we have some element $z\in L_i$ such that

\[\frac{y}{(1+\epsilon/2n)^i} \le z \le y\]

This is because at each step we have a $\frac{y}{1 + \delta} \le z \le y$, so
through induction we must have $\frac{y}{(1+\delta)^i} \le z \le y$.

\[y^*/z \le \left(1 + \frac{\epsilon}{2n}\right)^n\]

Because $y^*$ is the largest element in $P_n$ less than $t$ and $z$ is the
same but for $L_n$. We have

\[\left(1+\frac{\epsilon}{2n}\right)^n \le 1 + \epsilon\]

This follows from a lot of shit scattered all over the book (and some
excercises). This proves the approximation-factor.

For running time observe that the running time of the algorithm is polynomial
in the lengths of $|L_i|$ for all the lists. After trimming a such list we
have that two elements $z_i, z_{i+1}$ must have $z_{i+1}/z_i > 1+\epsilon/2n$.
A list therefore has at most as many values as this list:

\[\{0, 1, 1 + \epsilon/2n, (1 + \epsilon/2n)^2, \ldots, (1+\epsilon/2n)^k\}\]

where $k$ is the largest integer such that the expression is $\le t$. The list
thus has:

\[\log_{1 + \epsilon/2n} t + 2 = \frac{\ln t}{\ln(1 + \epsilon/2n)} + 2\]

elements. This is also $< \frac{3n\ln t}{\epsilon} + 2$ because
$0 < \epsilon < 1$.

This is polynomial in the input size which must certainly be greater than $n$
plus the amount of bits, $\lg t$ needed to represent $t$.

\end{proof}


\end{document}
