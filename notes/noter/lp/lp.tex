\documentclass[a4paper]{article}

%Packages
\usepackage[english]{babel}
\usepackage[ansinew]{inputenc}
\usepackage{amsmath, amssymb, amsthm}
\usepackage{cancel}
\usepackage[T1]{fontenc}
\usepackage{graphicx}
\usepackage[stable]{footmisc}
\usepackage{lastpage}
\usepackage{ulem}
\usepackage{listings}
\usepackage[usenames,dvipsnames]{color}
\usepackage{clrscode3e}
\usepackage{multirow}
\usepackage{hyperref}

%Listings
\definecolor{listingShade}{RGB}{245,245,245}
\lstset{ %
%language=PHP, %LANGUAGE HERE! :D
basicstyle=\ttfamily,
numbers=left,
numberstyle=\ttfamily\footnotesize,
stepnumber=1,
numbersep=5pt,
backgroundcolor=\color{listingShade},
showspaces=false,
showstringspaces=false,
showtabs=false,
frame=tb,
tabsize=8,
captionpos=b,
breaklines=true,
breakatwhitespace=false,
title=\lstname
}

%Theorems
\theoremstyle{definition}
\newtheorem{thm}{Theorem}[section]
\theoremstyle{definition}
\newtheorem{lem}{Lemma}[section]
%\theoremstyle{definition}
%\newtheorem{defi}{Definition}%[section]

%Math sets
\newcommand{\N}{\mathbb N} %the natural numbers
\newcommand{\Z}{\mathbb Z} %the integers
\newcommand{\Q}{\mathbb Q} %the rational numbers
\newcommand{\R}{\mathbb R} %the real numbers
\newcommand{\C}{\mathbb C} %the complex numbers

%Other commands
\newcommand{\ul}{\underline}
\newcommand{\ol}{\overline}
\newcommand{\mono}[1]{{\ttfamily#1}}

%Title stuff
\title{Advanced algorithms\\Linear Programming Notes}
\author{S�ren Dahlgaard}

\begin{document}

\maketitle

\section{Disposition}
\begin{enumerate}
\item What is a linear program?
\item Standard and slack forms
\item Pivoting
\item Simplex algorithm. Pivot until we're done.
\item Simplex terminates. A slack form is determined by $B$ (or $N$). Finite
    amount. Cycling can be avoided
\item Dual and primal. Weak duality
\item Intuition that optimal solutions will be equal.
\item Initial slack form.
\end{enumerate}


\section{Summary}

A linear program is where we want to maximize/minimize some linear function:

\[
\text{maximize } \sum_{j=1}^n c_jx_j
\]

subject to a set of linear constraints

\[
\text{subject to } \sum_{j=1}^m a_{ij}x_j \le b_i \quad i = 1..m
\]

We can also have $\ge$ or $=$.

\section{Standard and slack forms}
A linear program in standard form requires that all variables be positive
and all linear constraints be ``less than or equal''. That is:

\[
\text{max. } \sum_{j=1}^n c_jx_j
\]
\begin{align}
\text{st. } \sum_{j=1}^n a_{ij}x_j &\le b_i \quad \text{for } i = 1..m \\
x_j &\ge 0 \quad \text{for } j=1..n
\end{align}

We can write these as matrices: $A = (a_{ij})$ an $m\times n$ matrix.
$b = (b_i)$, $c = (c_j)$ and $x = (x_j)$. respectively $m$-vector and two
$n$-vectors. Then we want to maximize $c^Tx$ subject to $Ax\le b$ and $x\ge 0$.

We call a solution $\ol{x}$ feasible if it satisfies all constraints. An
optimal solution is a solution $\ol{x}$ which has maximum objective value
$c^T\ol{x}$ over all feasible solutions. Some linear programs may be unbounded.

To convert a linear program into standard form do the following:
\begin{enumerate}
\item To convert minimization to maximization, just multiply $c$ by $-1$.
\item For each variable $x_j$ without nonnegativity constraint. Replace it
    with $x_j' - x_j''$. We can set $x_j', x_j'' \ge 0$
\item Equality constraints can be replaced by two inequality constraints.
\item $\ge$ constraints can be replaced by $\le$ by multiplying with $-1$.
\end{enumerate}

\subsection{Slack form}
In slack form we want to replace all inequality constraints with equality
constraints, so we introduce variables

\[x_{n+i} = b_i - \sum_{j=1}^n a_{ij}x_j\]
\[x_{n+i} \ge 0\]

If these two equations hold then clearly the inequalities from the standard
form holds - and vice versa. We call this variable a slack variable because it
indicates how much slack there is between the left and right-hand side.

Example:

\begin{align}
z   &=     2x_1 - 3x_2 + 3x_3 \\
x_4 &=  7  - x_1 - x_2  + x_3 \\
x_5 &=  -7 + x_1 + x_2  - x_3 \\
x_6 &=  4  - x_1 + 2x_2 - 2x_3
\end{align}

We call the variables on the left-hand side of the equations
for the basic variables and denote their indices by the set $B$. The
variables on the right-hand side are called nobasic variables, denoted $N$.
This gives:

\begin{align}
z   &= v   + \sum_{j\in N}c_jx_j \\
x_i &= b_i - \sum_{j\in N}a_{ij}x_j \quad \text{for } i \in B
\end{align}

In the example we have:

\[B = \{1,2,3\}\]
\[N = \{4,5,6\}\]
\[A = \begin{pmatrix} -1 & -1 &  1 \\
                       1 &  1 & -1 \\
                      -1 &  2 & -2 \end{pmatrix}\]
\[b = \begin{pmatrix} 7 & -7 & 4 \end{pmatrix}^T\]
\[c = \begin{pmatrix} 2 & -3 & 3 \end{pmatrix}^T\]
\[v = 0\]


\section{SIMPLEX}

The idea is to rewrite the program every step until we have rewritten it in
a way such that the optimal solution is easy to obtain.

Simplex starts by converting the linear program into slack form. There are
a couple of concepts to simplex.

\begin{description}
\item [Pivoting] From one slack form we pick an ``entering variable''. This is
    picked at random from the nonbasic variables. It must have $c_j > 0$. Let
    this variable be $x_e, e\in N \land c_e > 0$. As we increase this variable
    the values of $x_j, j\in B$ might either increase, decrease or stay the
    same. If no $x_j$ decreases it is clear that we can increase $x_e$
    infinitely and the objective function is unbounded.

    We cannot allow any $x_j$ to become negative, as the solution won't be
    feasible then. We therefore pick the $i\in B$ that minimizes $b_i/a_{ie}$.
    We call this the leaving variable $l$. We now look at the constraint
\[
x_l = b_l - \sum_{j\in N}a{lj}x_j
\]
    and we exchange the roles of $x_l$ and $x_e$ which gives us:
\[
x_e = b_l/a_{le} - \sum_{j\in N + \{l\} - \{e\}} a_{lj}/a_{le}x_j
\]
    We then replace $x_e$ in all other $x_i, i\in B$ with this new expression.

    Finally we change the objection value $v = v + c_eb_e = v + b_l/a{le}$
\end{description}

In order to show that simplex returns either ``unbounded'' or a feasible
solution we use the following invariant:

\begin{enumerate}
\item The slack form is equivalent to the initial slack form
\item All $b_i\ge 0$
\item The basic solution is feasible
\end{enumerate}

\begin{description}
\item [Initialization:]
    \begin{enumerate}
    \item They are the same
    \item The assumption is that the initial is feasible, so this must be true.
    \item Is true because of (2).
    \end{enumerate}
\item [Maintenance:]
    We assume that there is a bounded variable $x_e$
    \begin{enumerate}
    \item Can be shown with simple linear algebra.
    \item Clearly $b_e = b_l/a_{le}$ is positive. We also know that
        $b_i/a_{ie} \ge b_l/a_{le}$. And that the new $b_i = b_i - a_{ie}b_e$.
        This clearly gives $b_i\ge 0$ (because $a_{le} > 0$).
    \item (2) causes this to be true because all $x_j, j\in N$ are $0$ and all
        $b_i, i\in B$ are $\ge 0$.
    \end{enumerate}
    The case where a variable is unbounded causes the algorithm to terminate.
\item [Termination:]
    If the algorithm terminates because there is no positive coef. in the
    objective function we have from maintenance that the solution is feasible.

    If we find an unbounded variable, we want to show that
\[
\ol{x}_i =
\begin{cases}
\infty                             & \text{if } i = e \\
0                                  & \text{if } i\in N - \{e\} \\
b_i - \sum_{j\in N} a_{ij}\ol{x}_j & \text{if } i\in B
\end{cases}
\]
    We know that all $b_i\ge 0$. We also know that all $x_i$ in $N$ are
    nonnegative. We need to see that $x_i = b_i - a_{ie}x_e$ is nonnegative.
    Because $a_{ie} \le 0$ (Other wise it would be bounding for $x_e$) this is
    okay.

    This gives an objective value of $\infty$.
\end{description}

We now need to show that Simplex terminates, and that the returned solution is
optimal. We also need to show how to find an initial solution.

\subsection{Termination}
There is a problem. No pivoting will decrease the objective value, but it
might not increase. See example:

\[z = 8 + x_3 - x_4\]
\[x_1 = 8 - x_2 - x_4\]
\[x_5 = x_2 - x_3\]

We have to pick $x_3$ as $x_e$ and $x_5$ as $x_l$ which gives no increase in
pivot value. This also means that SIMPLEX can cycle.

First let's show that $N$ and $B$ uniquely define the slack form. We use that
if we have
\[\sum_{j\in I}\alpha_jx_j = \gamma + \sum_{j\in I}\beta_jx_j\]
for any setting of $x_j$'s. Then $\gamma = 0$ and $\alpha_j = \beta_j$. This
is easy to show (set $x_j = 0$ then $\gamma = 0$ and so on...)

For any standard form $(A, b, c)$ and a set of basic variables $B$ the slack
form is uniquely defined.

\begin{proof}
Assume two different slack forms have the same basic/nonbasic variables.

Because all slack forms are equivalent we can subtract the two forms from each
other, giving:

\[
0 = (b_i - b_i') - \sum_{j\in N}(a_{ij} - a_{ij}')x_j\quad \text{for } i\in B
\]
or
\[
\sum_{j\in N}a_{ij}x_j = (b_i - b_i') + \sum_{j\in N}a_{ij}'x_j
\]
We know that this makes the form equivalent. Same goes for the objective
function.
\end{proof}

This means that SIMPLEX will always terminate unless it cycles. Because we
can only chose $m$ basic variables out of the $n+m$ variables in $n+m C m$
ways simplex will return in that amount of iterations unless it cycles. We
can make it not cycle by always picking the entering variable with lowest
index (bland's rule).


\section{Duality}
We want to show that simplex returns an \textit{optimal} solution.

Given a primal linear program in standard form we create the dual linear
program:

\[\text{min. } \sum_{i=1}^m b_iy_i\]
\[\text{s.t. } \sum_{i=1}^m a_{ij}y_i \ge c_j \quad \text{for } j = 1..n\]
\[y_i\ge 0 \quad \text{for } i = 1..n\]

If we have feasible solutions $\ol{x}, \ol{y}$ we have:

\[\sum_{j=1}^n c_j\ol{x}_j \le \sum_{i=1}^m b_i\ol{y}_i\]

\begin{proof}
\begin{align}
\sum_{j=1}^nc_j\ol{x}_j &\le
    \sum_{j=1}^n\left(\sum_{i=1}^m a_{ij}\ol{y}_i\right)\ol{x}_j \\
    &= \sum_{i=1}^m\left(\sum_{j=1}^n a_{ij}\ol{x}_j\right)\ol{y}_i \\
    &\le \sum_{i=1}^m b_i\ol{y}_i
\end{align}
\end{proof}

From this it is clear, that if the objective value of the primal and dual
are the same, then they are optimal.

If we have a slack form returned by simplex of the primal:
\[z = v' + \sum_{j\in N}c_j'x_j\]
\[x_i = b_i' - \sum_{j\in N}a_{ij}'x_j \quad \text{for }i\in B\]

we can produce an optimal dual solution by setting

\[
\ol{y}_i =
\begin{cases}
-c'_{n+i} & \text{if } (n+i)\in N \\
0 & \text{otherwise}
\end{cases}
\]

If SIMPLEX returns $\ol{x}$ and the last slack form $N, B, c$. Let $\ol{y}$ be
defined as stated above. Then the $\ol{x}$ is optimal for the primal and
$\ol{y}$ is optimal for the dual.

Because all $c'_j\le 0, j\in N$ in the last slack form, if we set
$c'_j =0, j\in B$. We can write:

\begin{align}
z &= v' + \sum_{j\in N}c'_jx_j \\
  &= v' + \sum_{j = 1}^{m+n} c'_jx_j
\end{align}

Because all $x_j = 0, j\in N$ in the basic solution we have that $z = v'$.

Because of the equality of slack forms we have
\[\sum_{j=1}^nc_jx_j = v' + \sum_{j=1}^{n+m}c'_jx_j\]
(Note that $c'_j = 0, j\in B$).
We therefor have:

\begin{align}
\sum_{j=1}^nc_j\ol{x}_j &= v' + \sum_{j=1}^{n+m}c'_j\ol{x}_j \\
&= v' + \sum_{j=1}^nc'_j\ol{x}_j + \sum_{i=1}^m c'_{n+i}\ol{x}_{n_i} \\
\end{align}

We can replace $c'_{n+i}$ with $-\ol{y}_i$. And $\ol{x}_{n+i}$ are the slack
variables, so these can be replaced with their formula in the original slack:

\[
= v' + \sum_{j=1}^nc'_j\ol{x}_j + \sum_{i=1}^m (-y_i)\left(b_i - \sum_{j=1}^n a_{ij}\ol{x}_j\right)
\]

Regrouping gives us

\[
= v' + \sum_{j=1}^nc'_j\ol{x}_j - \sum_{i=1}^m b_i\ol{y}_i +
    \sum_{j=1}^n\sum_{i=1}^m (a_{ij}\ol{y}_i)\ol{x}_j
\]

We then get

\[
\sum_{j=1}^n c_j\ol{x}_j = \left(v' - \sum_{i=1}^m b_i\ol{y}_i\right)
    + \sum_{j=1}^n\left(c'_j + \sum_{i=1}^m a_{ij}\ol{y}_i\right)\ol{x}_j
\]

This is of the form we proved earlier, so we have

\[v' - \sum_{i=1}^mb_i\ol{y}_i = 0\]
\[c'_j + \sum_{i=1}^ma_{ij}\ol{y}_i = c_j \quad \text{for } j = 1..n\]

The first of these two give that $v' = \sum_{i=1}^m b_i\ol{y}_i$, so the
objective values of the primal and dual are the same. We now only to show,
that $\ol{y}$ is feasible.

Because $c'_j\le 0$ (or simplex would continue). From the equations above we
have

\[c_j = c'_j + \sum_{i=1}^m a_{ij}\ol{y}_i \le \sum_{i=1}^m a_{ij}\ol{y_i}\]

We therefore have that the feasible solution returned by SIMPLEX is optimal.


\section{Finding the initial slack form}
The idea is to create a new linear program $L_{\text{aux}}$:

\[\text{maximize } -x_0\]
\begin{align*}
\text{subject to } \sum_{j=1}^n a_{ij}x_j - x_0 &\le b_i \\
x_j &\ge 0
\end{align*}

This has objective value $0$ if and only if $L$ is feasible.

\begin{proof}
$\Rightarrow$: If $\ol{X}$ is feasible solution to $L$, then we can set $x_0=0$
and all the constraints are satisfied. This will give $z=0$, but this is
obviously optimal.

$\Leftarrow$: If $x_0 = 0$ then the remaining values make up a solution to $L$.
\end{proof}

The idea is, that we can perform one pivot with $x_0$ as the leaving
variable to get a feasible solution to $L_{\text{aux}}$. Then we can use
SIMPLEX to get the optimal solution. If it's objective value is $0$, we're
good.

\begin{proof}
If we perform a pivot with $0$ entering and $i$ such that $b_i$ is minimal
leaving, then the base solution is feasible. This is true becaus $x_0$ has
the same coefficient in alle the constraints, so picking the minimum $b_i$
is enough.

Simplex does the rest.
\end{proof}

\end{document}
