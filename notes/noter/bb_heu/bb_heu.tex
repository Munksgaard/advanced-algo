\documentclass[a4paper]{article}

%Packages
\usepackage[english]{babel}
\usepackage[ansinew]{inputenc}
\usepackage{amsmath, amssymb, amsthm}
\usepackage{cancel}
\usepackage[T1]{fontenc}
\usepackage{graphicx}
\usepackage[stable]{footmisc}
\usepackage{lastpage}
\usepackage{ulem}
\usepackage{listings}
\usepackage[usenames,dvipsnames]{color}
\usepackage{clrscode3e}
\usepackage{multirow}
\usepackage{hyperref}

%Listings
\definecolor{listingShade}{RGB}{245,245,245}
\lstset{ %
%language=PHP, %LANGUAGE HERE! :D
basicstyle=\ttfamily,
numbers=left,
numberstyle=\ttfamily\footnotesize,
stepnumber=1,
numbersep=5pt,
backgroundcolor=\color{listingShade},
showspaces=false,
showstringspaces=false,
showtabs=false,
frame=tb,
tabsize=8,
captionpos=b,
breaklines=true,
breakatwhitespace=false,
title=\lstname
}

%Theorems
\theoremstyle{definition}
\newtheorem{thm}{Theorem}[section]
\theoremstyle{definition}
\newtheorem{lem}{Lemma}[section]
%\theoremstyle{definition}
%\newtheorem{defi}{Definition}%[section]

%Math sets
\newcommand{\N}{\mathbb N} %the natural numbers
\newcommand{\Z}{\mathbb Z} %the integers
\newcommand{\Q}{\mathbb Q} %the rational numbers
\newcommand{\R}{\mathbb R} %the real numbers
\newcommand{\C}{\mathbb C} %the complex numbers

%Other commands
\newcommand{\ul}{\underline}
\newcommand{\ol}{\overline}
\newcommand{\mono}[1]{{\ttfamily#1}}

%Title stuff
\title{Advanced algorithms\\Branch \& Bound + Metaheuristics Notes}
\author{S�ren Dahlgaard}

\begin{document}

\maketitle

\section{Disposition}
\begin{enumerate}
\item Motivation - NP-hard problem.
\item Functions $f$ and $g$
\item Sets $P$ and $S$.
\item General procedure
\begin{enumerate}
\item Keep a set of unexplored subproblems. Each with a bound $g(P_i)$.
\item View this as a tree and explore subproblems by branching on them
\item Requirements of $g$: $g(P_i) \le f(P_i)$ and parent-child.
\item Lazy vs Eager
\end{enumerate}
\item Bounding function
\begin{enumerate}
\item $\min_{x\in P} f(x)$ - Relax some requirements.
\item $\min_{x\in S} g(x)$ - Modify objective function
\item Combine these. We can parameterize. Langrangean relaxation.
\end{enumerate}
\item Selection strategy - BeFS, DFS, IDDFS, etc.
\item Branching strategy - TSP example
\begin{enumerate}
\item Create permutation picking next vertex each time
\item Create subset deciding if $e_i$ is in or not.
\end{enumerate}
\item TSP 1-tree example
\item Time analysis of worst case
\begin{enumerate}
\item k-SAT.
\item Branching vector
\item Branching factor. $\tau(1,2,\ldots,q)$.
\end{enumerate}
\item \textit{Initial incumbent - metaheuristics}
\begin{enumerate}
\item gradient search all of them.
\item Simulated annealing
\item Hill climbing
\item Tabu search
\end{enumerate}

\end{enumerate}

\section{B\&B - 1}
\subsection{Motivation}
We have already seen that some problems may very likely be intractable to try
and solve completely (NP-Hard problems). We have also seen that we can some
times approximate these problems very well, however sometimes this is not
possible (TSP) or the approximation is simply not good enough.

One of the most widespread ways to find optimal solutions is using branch and
bound.

\subsection{Procedure}
Let $S$ be the set of feasible solutions to a problem $A$ and let $f$
be a function that associates a solution with its objective value.
\begin{description}
\item [Example] If $A$ is TSP, then $S$ would be all permutations of $V$ and
    $f$ would be a function summing the weights of the edges.
\end{description}

We also define $P$ as a set of all potential solutions and a $g$ which is a
bounding function.

\begin{description}
\item [Example] If $A$ is TSP then $P$ could be the family of all sets of
    subtours in $G$. The bounding function $g$ will be described later.
\end{description}

The idea is to keep a set of unexplored solutions and then at each step of
the algorithm we split the set in subsets (branch) and compute a bound
on the best solution in each of these sets (bound). We also keep a ``current
best'' or incumbent, so if one of the subsets $S_i$ has a bound worse than the
incumbent, then we discard $S_i$.

In order to do this we view the problem as a tree with each node representing
a set of potentional solutions $P_i$. Each $P_i$ thus represents a subproblem
of the original problem by imposing extra constraints (eg. ``edge $e_i$ must
be in the tour'' for TSP). If $P_j$ is the child of $P_i$ then $P_i$ is a
subproblem derived from $P_i$.

We call all nodse $P_i$ in the tree for live. If some node has a worse bound
than the incumbent we kill/discard/fathom it. The same goes for nodes that
cannot lead to feasible solutions.

In order to do this we normally require some things of the bounding function
$g$:

\begin{itemize}
\item $g(P_i) \le g(P_i)$ for all nodes in the tree.
\item $g(P_i) = f(P_i)$ for all leaves in the tree.
\item $g(P_i)\le g(P_j)$ if $P_j$ is a child of $P_i$.
\end{itemize}

\subsection{Lazy vs Eager}
There are two ways to build the tree. The eager way we create the bound each
time we create a node (when branching). In the lazy way we wait until we are
picking the node to try and branch on, so we pick each node based on the bound
of its parent, then evaluate the node's bound and see if it's better than the
incumbent. If it is, we branch.

If we use BFS or DFS it can be a good idea to use the lazy method, because the
order in which we visit the nodes isn't determined by their bounds. If we use
a BeFS it can still be done with lazy, but it doesn't make much sense.

\subsection{The three elements of B\&B}

As can be seen we use three elements in this approach:

\begin{description}
\item [Bounding function:] The bounding function is the most important part
    of the B\&B algorithm. If we use a weak bounding algorithm we will end
    up traversing huge parts of the search tree (there will be many critical
    nodes). If we use a strong bounding function however we will only visit
    very few of the nodes in the search tree.\\
    We often have a tradeoff of precision versus time when picking a bounding
    function. For instance a bounding function satisfying $g(x) = \min f(x)$
    over its children will be NP-Hard itself.\\
    There is generally three ways of making lower bounds:
    \begin{itemize}
    \item $\min_{x\in P} f(x)$. This means that we relax some of the
        requirements on the problem and calculate the optimal solution. For
        instance we can get a lower bound for the 0/1-Knapsack by calculating
        the value of the fractional knapsack.
    \item $\min_{x\in S} g(x)$. This way we calculate the bound for the
        problem itself, but we modify the objective function. If we do this
        we will not likely have $f(x) = g(x)$ for leaves. An example of this
        is to simply use the constrains applied to the $P_i$ we're calculating
        the bound for. For instance using TSP with a problem $P_i$ defining
        the extra constraints that some edges MUST be in the tour, then we
        simply sum the weight of these edges.
    \item We might also combine the two to get $\min_{x\in P} g(x)$. This might
        seem weaker, but we can parameterize it to find a good parameter. An
        example of this is described below.
    \end{itemize}
\item [Selection strategy:] This defines the order in which we pick the nodes
    in our search tree.
    \begin{description}
    \item [BFS] Simply go through the search tree one level at a time. This is
        however not very smart because it requires a lot of memory and
        potentially a lot of time. Only use this if we know that the optimal
        solution is very close to the root (in the tree). \textit{Queue}
    \item [BeFS] Pick the live node with the lowest bound first. This way we
        only visit critical nodes - nodes $x\in S$ with $g(x)\le f(x')$ for
        optimal $x'\in S$. This can however use just as much memory as a BFS
        in the worst case and in general uses a lot of memory. \textit{Priority
        queue}
    \item [DFS] Simply pick a live node of max depth in the tree. This ensures
        that at most $b$ nodes per level will be live, where $b$ is the maximal
        branch factor. So the memory use is $O(bd)$ where $d$ is the depth of
        the search tree. DFS may however visit a lot of non-critical nodes.
    \item [IDDFS] Iterative deepening DFS. Combine DFS with a depth limit. This
        way we explore all levels at a time like in BFS, but we also get the
        space complexity of DFS. The running time turns out to be the same. In
        most cases the earlier searches (ie. smaller depths) give good bounds
        for use in the later searches.
    \end{description}
    We might also combine DFS with an ordering of the nodes of a level, so
    instead of just picking any node of maximum depth, we pick the ``best''
    node of maximum depth.
\item [Branching rule:]
\end{description}

We also want a good initial incumbent.

\subsection{Good TSP bound}
Use a \#1-tree:

\begin{enumerate}
\item Pick a vertex $v$ and remove it from the graph, then create a MST of the
    remaining graph.
\item Add the two smallest-weight edges incident $v$ to the graph
\item The result is a graph with $|V|$ edges that might not be a tour.
\end{enumerate}

This is obviously a lower bound on the optimal TSP tour. If the returned
graph is a tour then it must be optimal.

We now have two choices:

\begin{enumerate}
\item Improve the bound we found
\item Branch
\end{enumerate}

The idea is that some vertices will have more than $2$ incident edges and thus
some will have less than $2$. If we choose to branch on a vertex like that we
make $\text{degree}(v)$ ned nodes for each of those nodes with $>2$ incident
edges. For each of these nodes we remove an associated edge from the graph.
This will however cause some subproblems $P_i$ to overlap.

If we want to imporve the bound, we define a number $\pi(v)$ for each vertex
$v\in V$ to be $\pi(v) = \text{degree}(v) - 2$ (for the degree of $v$ in $T$).
For each edge $(u,v)$ we now define a modified cost:
\[c'(u,v) = c(u,v) + \pi(u) + \pi(v)\]
Note that we have
\[\sum_{v\in V} \pi(v) = \sum_{v\in V} \text{deg}(v) - 2|V| = 0\]
Because we have exactly $n$ edges. Any hamilton tour will thus have cost:
\begin{align}
\sum_{e\in H} c(u,v) + \pi(u) + \pi(v) &=
    \sum_{e\in H} c(u,v) + \sum_{e\in H} \pi(u) + \pi(v) \\
&=  \sum_{e\in H} c(u,v) + \sum_{v\in V} 2\pi(v) \\
&= \sum_{e\in H} c(u,v)
\end{align}
Where $e\in H = (u,v)$ is some edge in the hamilton cycle. Thus we have that
all hamilton tours under this new constraint have the same value, though
1-trees will likely cost more - giving a better bound.


\subsection{Branching rule}
A branching rule is how we branch. For instance the above strategy of creating
$\text{deg}(v)$ new nodes for a node $v$ is a branching strategy.

Convergence is ensured if subproblems $P_j$ of $P_i$ are smaller and there's a
finite branching factor $b$.

Other branching strategies could be:

\begin{itemize}
\item For TSP start with vertex $1$. Each time we branch create a branch for
    all vertices not yet in the given tour. so first we create the paths:
    $(1,2), (1,3), (1,4), \ldots, (1,n)$ and then for subproblem $(1,4)$ we
    will get the problems $(1,4,2), (1,4,3), (1,4,5), \ldots, (1,4,n)$. \\
    In this case we won't get identical subproblems.
\item For TSP look at the edges of $E$ in some order and for each branching
    make a node whether to include edge $e_i$ or not. This gives us a binary
    tree but with much bigger depth than the previous one ($|E|$ versus $|V|$).
\end{itemize}

Neither of these two strategies give overlapping subproblems though the 1-tree
branching does. This is not incorrect, but just less efficient.

\subsection{Initial incumbent}
If we can produce a good initial incumbent we can limit the amount of nodes
searched with for instance DFS or BFS. This is often done with metaheuristics.


\section{Branch \& Bound - 2}
For $L\in NP$ we can state the problem as:
\[
\text{Given } x \text{ find } y \text{ such that } |y| \le |x|^c \land A(x,y)
\]

(See NP notes). Thus if we enumerate all $y$ with $|y|\le |x|^c$ we can find
the solution. 
\begin{description}
\item [Subset problems] For each element $x$ we can have either $x\in S'$ or
    $x\notin S'$. This gives $2^{|S|}$ possibilities.
\item [Permutation problems] Given a set of size $n$ there is $n!$
    permutations.
\item [Partition problems] We need to partition a set of $n$ elements into
    different groups from size $1$ to $n$. There is $n^n$ possibilities.
\end{description}

\subsection{Maximum Independent Set}
We want to pick a set $I\subseteq V$ such that for all vertices $u,v\in V$
we have $(u,v)\notin E$. The naive version would thus take $O(2^n)$ time.

The idea is that for each vertex $v$ we must have either $v\in I$ or one of
its neighbours $u\in N[v]$ must be in $I$ (otherwise we could add $v$). Thus
we pick a vertex $v$ of minimal degree, then we make a new node in the search
tree for each node $u$ in its neighbourhood and assume that this node has to
be in the set for that given subproblem. Because one of those vertices MUST be
in $I$ it is clear that the algorithm produces the correct answer. In general
correctness proofs for branching algorithm follow easily from the fact that
the algorithm examines all possibilities.

\textbf{Running time analysis:} The running time of a B\&B algorithm is a
function on the maximal size of the search tree. The algorithm picks a vertex
of minimum degree and recurses on all vertices in its neighbourhood. Thus we
get the following recursion formula:

\[
T(n) \le 1 + T(n - d(v) - 1) + \sum_{i=1}^{d(v)} = T(n - d(v_i) - 1)
\]

That is: 1 node for the current. One subproblem for $v$ where we have removed
$d(v) + 1$ nodes (all nodes in $N[v]$) and one subproblem for each
$v_i\in N[v]$ where we remove all nodes in $N[v_i]$. Because $d(v_i) \ge d(v)$
we can write:

\[
T(n) \le 1 + (d(v) + 1)T(n - d(v) - 1)
\]

Let $s = d(v) + 1$ We then have

\[
T(n) \le 1 + sT(n-s) \le 1 + s + s^2 + \ldots + s^{n/s}
\]

Because the recursion $T(n-s)$ goes down $n/s$ levels. So we have:
\begin{align}
T(n) &= 1 + sT(n-s) \\
&= 1 + s(1 + sT(n-2s)) \\
&= 1 + s + s^2T(n-2s) \\
&= 1 + s + s^2(1 + sT(n-3s)) \\
&= 1 + s + s^2 + s^3(n-4s) \\
&= \ldots
\end{align}

This is a geometric series with the value

\[
\frac{1 - s^{n/s + 1}}{1-s} = O(s^{n/s}) = O(3^{n/3})
\]

Note. $3$ is the integer value that maximizes the formula, otherwise $e$ would
be better. Also note that we ignore the polynomial factors associated with
each node.


\subsection{General time analysis}
A general way of analyzing branching algorithms is an open problem. In general
we want to find the smallest $\alpha$ such that the running time is
$O(\alpha^n)$.

To find this we use a branching vector: Let $r\ge 2$ be the amount of branches
and let $t_i, i = 1..r$ be integers $n\ge t_i>0$ such that branch number $i$
is a subproblem of size $n - t_i$. Then we call $b = (t_1, \ldots, t_r)$ the
branching vector. Because $r\ge 2$ we can do with just looking at the amount
of leaves in the search tree because they make up at least half the tree. So
we get:

\[
T(n) \le T(n-t_1) + T(n-t_2) + \ldots + T(n-t_r)
\]

A solution to this is of the form $c^n$ with $c$ being the complex root of:

\[
x^n - x^{n-t_1} - \ldots - x^{n-t_r} = 0
\]

We call this $c$ for $\alpha$ or the branching factor. We denote this
$\tau(t_1,\ldots,t_r)$ for a branching vector. For instance
$\tau(2,2) = \sqrt(2)$. With our restrictions on the branching vector we have:

\begin{enumerate}
\item $\tau(t_1,\ldots,t_r) > 1$
\item $\tau(t_1,\ldots,t_r) = \tau(t_{\pi(1)}, \ldots, t_{\pi(r)})$ for some
    permutation $\pi$ of $1..r$.
\item $\tau(t_1,\ldots,t_r) < \tau(t_1',\ldots,t_r)$ if $t_1 > t_1'$.
\end{enumerate}

We also have

\begin{enumerate}
\item $\tau(k,k) \le \tau(i,j)$ for $i+j = 2k$.
\item $\tau(i,j) > \tau(i+\epsilon, j-\epsilon)$ for $i<j$ and
    $0 < \epsilon < (j-i)/2$.
\end{enumerate}

A problem may have several branching vectors, so we have to find the ``worst''
one when calculating the upper bound. We can also add branching vectors, so
if we branch $(i,j)$ and immediately branch $(k,l)$ on the $i$-branch. Then we
get the branching vector $(i+k, i+l, j)$ - Remember that we ignore the
polynomial cost of the $i$-node.


\subsection{k-SAT}
If there is $|L|$ variables we can have at most $2|L|$ literals (both the
variable and its negation). Each clause has at most $k$ literals, so in total
there can be at most
\[
m\le \sum_{i=1}^k \begin{pmatrix} 2n \\ i \end{pmatrix}
\]
unique clauses.

We use the following branching rule. Let $F$ be the formula we are checking.
Let $c$ be some clause in $F$. Let $(l_1, \ldots, l_q)$ be the $q\le k$
literals in $c$. Create the following $q$ branches:

\begin{enumerate}
\item $l_1 = \const{True}$
\item $l_1 = \const{False}, l_2 = \const{True}$
\item $\ldots$
\item $l_1 = l_2 = \ldots = l_{q-1} = \const{False}, l_q = \const{True}$
\end{enumerate}

Let $T(n)$ be the running time for $n$ literals, then we get the branching
vector $(1, 2, \ldots, q)$. It turns out that $\tau(1,2,\ldots, q)$ is the
largest real root of:

\[x^{q+1} - 2x^q - 1 = 0\]

For each $F$ we thus have a branching rule for each clause. The clause with
fewest literals will give the best branching.

We will now show that we can do this in a way that always branches on a
clause with $q \le k-1$ except for the first time. To this we use something
called ``autarks''. An autark is an assignment of truthvalues $t$ such that
for any clause $c\in F$ if $t$ sets one literal of $c$ then it MUST set one
of the literals to true. That is. All clauses that $t$ reduces, it satisfies.
If $F'$ is the subproblem remaining when setting $F$ to $t$, then we have that
whether $F'$ is satisfiable or not is independant of $t$, so we have
$F'$ is satisfiable iff $F$ is satisfiable. Thus if we have added enough
constraints to create an autark we don't need to branch, but rather recurse
(this is a reduction rule rather than a branching rule).
Otherwise we know that there must be a clause for which our $t$ has set a
literal to false, but none to true, so it must have $q\le k-1$ literals. Thus
for 3-CNF we need to solve:

\[x^3 - 2x^2 - 1 = 0\]

which gives a running time less than $O(1.6181^n)$.


\subsection{Maximum independant set}
If minimum degree of $v\in G$ is $\ge 3$ we pick a vertex of minimum degree.
Otherwise we pick a vertex of maximum degree. We then either branch or reduce.

\begin{description}
\item [Domination rule] For any $v,w\in V$ if $N[v]\subseteq N[w]$ we can make
    a MIS without $w$. This is obvious. Consider $w\in I$ then no neighbours
    of $v$ can be in $I$, thus $I\setminus \{w\} \cup \{v\}$ is an independant
    set of the same size.
\item [2.6] If no MIS contains $v$ then ALL MIS contain at least two vertices
    from $N[v]$. Assume for contradiction that an MIS doesn't contain anything
    from $N[v]$, then we could add $v$. If it only contains one element from
    $N[v]$ we could remove that element and add $v$. Contradiction
\item [2.7] Let $M(v)$ be all vertices at distance $2$ from $v$ such that
    $N[v]\setminus N[w]$ is a clique (this is called mirror vertices). Then
    a MIS either contains $v$ or none of its mirrors.\\
    Proof: If $w\in M(v)$ we have $N[v]\setminus N[w]$ is a clique and thus
    only one element from it can be in a MIS. Thus there must be one element
    from $N[w]$ in the MIS.
\item [Simplical] If $N[v]$ is a clique then  we have
    $\alpha(G) = 1 + \alpha(G\setminus N[v])$, where $\alpha(G)$ is the size
    of a MIS for $G$.
\item [Disconnected graphs] Let $G$ be a disconnected graph, then we can
    calculate the MIS for each of the connected components and add them.
\item $\ldots$
\end{description}

The idea is that the worst case branching is $(1,6)$ and that one is guaranteed
to branch nicely on the $1$-branch, giving a better branching.


\section{Metaheuristics}
The metaheuristics we look at are almost all based on gradient ascent from
mathematics. In gradient ascent we want to maximize the value of $f(x)$ over
all possible values of $x$. We do not necessarily have a way to compute $f(x)$,
but we know $f'(x)$ - the slope of $f(x)$. We then keep on computing
\[x = x + \alpha \nabla f'(x)\]
until we find the optimal value of $x$. Maybe we even know $f''(x)$ also. Then
we use Newton's method and add $\alpha (\nabla f'(x))/f''(x)$.

\subsection{Hill-climbing}
However in algorithms we are rarely able to compute $f'(x)$ and even more
rarely $f''(x)$.

We do however often have a way to assess the value of a potential solution $x$.
For instance we can sum the edges of a hamilton cycle to assess a possible
TSP tour. We are also often able to tweak a solution a little bit (eg. swapping
the spots of two vertices in a hamilton cycle given that the edges exist).

The idea of hill climbing is to tweak the current solution a little bit and
use the new solution if it is better. If we use steepest-ascend hill climbing
we make $n-1$ tweaks to a solution and pick the best of these $n$ total
potential solutions.

Tweaking the solution much will give us better exploration while it might
accidentally overshoot the peaks of the different hills and thus never find
an optimal solution.

\subsection{Global optimization}
We often want to make a global optimization algorithm. That is: Given enough
time, the algorithm \textit{will} find the global optimal solution. Just
using hill climbing we cannot be sure, that we tweak the algorithm enough to
avoid local maxima. We might not tweak enough to jump from one hill to the
next.

We can fix this by applying random restarts to the hill climbing algorithm.
Say when we start climbing up one hill we're given some random amount of time
to do this. When this time is up we start at a new random point and climb up
this new hill.

For some graphs hill climbing will do nice and for some it won't. We say that
a graph where a small change in $x$-value causes a small change in $y$-value
is smooth. However this is not enough for hill climbing to do well. We must
also have an informal gradient. For instance some graphs might be very smooth
but unless you land in 2\% or the graph you will be lead up to a suboptimal
solution. (page 18/20)

In general we can make a heuristic global by the following 4 ideas:

\begin{description}
\item [Adjust tweaking] Occasionally make large tweaks.
\item [Adjust selection] Occasionally go down hills.
\item [Start over] Occasionally start over at a random location
\item [Large sample] Run the algorithm in parallel.
\end{description}

\subsection{Simulated annealing}
This is a modified version of hill climbing. We still pick tweaked solutions
$R$ over the current solution $S$ if $f(R) < f(S)$. But if not we might still
pick $R$ according to a parameter $t$. We do this with probability:
\[P(t,R,S) = e^{\frac{f(R) - f(S)}{t}}\]
We start out with $t$ being a high number and decrease it each iteration
until it is $0$.

In beginning this simulates a ``random walk'' and as $t$ decreases it looks
more and more like a hill climbing algorithm. If we set $t$ large enough it
is clear that this is a global algorithm.

\subsection{Tabu search}
Tabu search uses a list of recently considered solutions that we are not
allowed to visit again for some time. So when we walk up a hill we HAVE to
walk down the other side because we cannot return.

One way of doing this is to keep a list $L$ of some length $l$, so when we
find a new solution $R$ we add it to $L$ and if $|L| > l$ we remove the oldest
element.

Note that if we use a real-valued space it is not very likely that we get the
same solution twice, so Tabu search is best in discrete space. Also in other
cases it might be very easy to stick around the same hill long enough to
crawl up it again (eg. a very large search space).

A modified Tabu search stores tweaks rather than solution. For instance in TSP
we might swap the spots of $v,u$ in the permutation. We then add $v,u$ to $L$,
and then we aren't allowed to swap their positions again for a while. We can do
this eg. by a hash-table where we map each tweak (key) to a timestamp. We must
provide this table to the method that creates new potential solutions. This
method is a bit different from the rest because we don't consider a solution as
an atomic entity, rather it consists of different features that we can taboo.


\subsection{Tweaking and neighbourhoods}
When tweaking one way is to define a neighbourhood function $N(s)\subseteq S$
for each element $s\in S$. For instance in TSP a neighbourhood function could
be to swap any two vertices in the path, leading to a $O(n^2)$ size
neighbourhood.

A neighbourhood function also relies on the way we represent a solution. For
instance if we represent a solution as a permutation of $n$ vertices it might
be easy to swap two vertices. If we represent a solution as a list of
edges it might not.

Larger neighbourhoods take longer time to search (in eg. steepest-ascent
hill climbing), but they also usually lead to better local optima.

It is also important that the neighbourhood function makes it possible to
reach all solutions by going through some sequence of solutions. Note that
even though there might be an obvious way from some solution $s$ to $s'$ using
the neighbourhood function it doesn't necessarily go through feasible
solutions, so the way might not be real.


\end{document}
