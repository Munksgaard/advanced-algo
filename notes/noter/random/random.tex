\documentclass[a4paper]{article}

%Packages
\usepackage[english]{babel}
\usepackage[ansinew]{inputenc}
\usepackage{amsmath, amssymb, amsthm}
\usepackage{cancel}
\usepackage[T1]{fontenc}
\usepackage{graphicx}
\usepackage[stable]{footmisc}
\usepackage{lastpage}
\usepackage{ulem}
\usepackage{listings}
\usepackage[usenames,dvipsnames]{color}
\usepackage{clrscode3e}
\usepackage{multirow}
\usepackage{hyperref}

%Listings
\definecolor{listingShade}{RGB}{245,245,245}
\lstset{ %
%language=PHP, %LANGUAGE HERE! :D
basicstyle=\ttfamily,
numbers=left,
numberstyle=\ttfamily\footnotesize,
stepnumber=1,
numbersep=5pt,
backgroundcolor=\color{listingShade},
showspaces=false,
showstringspaces=false,
showtabs=false,
frame=tb,
tabsize=8,
captionpos=b,
breaklines=true,
breakatwhitespace=false,
title=\lstname
}

%Theorems
\theoremstyle{definition}
\newtheorem{thm}{Theorem}[section]
\theoremstyle{definition}
\newtheorem{lem}{Lemma}[section]
%\theoremstyle{definition}
%\newtheorem{defi}{Definition}%[section]

%Math sets
\newcommand{\N}{\mathbb N} %the natural numbers
\newcommand{\Z}{\mathbb Z} %the integers
\newcommand{\Q}{\mathbb Q} %the rational numbers
\newcommand{\R}{\mathbb R} %the real numbers
\newcommand{\C}{\mathbb C} %the complex numbers

%Other commands
\newcommand{\ul}{\underline}
\newcommand{\ol}{\overline}
\newcommand{\mono}[1]{{\ttfamily#1}}

%Title stuff
\title{Advanced algorithms\\Randomized Algorithms Notes}
\author{S�ren Dahlgaard}

\begin{document}

\maketitle

\section{Disposition}
\begin{enumerate}
\item Motivation - Quicksort example
\item Average-case analysis
\item Randomized algorithms. Ensuring average-case
\item Indicator random variables.
\begin{enumerate}
\item $E[X_H] = \text{Pr}\{H\}$.
\item Linearity of expectation
\end{enumerate}
\item Shuffling arrays to randomize things
\begin{enumerate}
\item Slow shuffle - easy proof
\item Knuth shuffle. Loop invariant: $(n-i+1)!/n!$
\end{enumerate}
\item Examples
\begin{enumerate}
\item Quicksort - Amount of comparisons. $Z_{ij}$.
\end{enumerate}
\end{enumerate}


\section{Summary}

Usually when analyzing algorithms we are only interested in the worst-case
time complexity. However in some cases we might want info about the
average-case. For instance we know that all splits on the form
$(\alpha, 1-\alpha)$ of quicksort result in logarithmic time. What is the
average-case?

\subsection{Probabilistic analysis}
When analyzing the average-case we want to make some assumptions about the
distribution of input. For instance assume that each ordering of candidates
is equally likely in the hiring problem (see below).

\subsection{Randomized algorithms}
In order to make assumptions about the distribution we might sometimes have
to randomize the input. For instance a very common case for sorting is
already sorted sequences or nearly sorted sequences. In these cases
quicksort performs bad, so what we do is that we randomize the algorithm such
that no specific input will ellicit worst-case behaviour.

Company vs company analogy.

\subsection{Indicator random variables}
In our probabilistic analysis we use variables known as indicator random
variables. For an event $A$ we define

\[
I\{A\} =
\begin{cases}
1 & \text{if $A$ occurs} \\
0 & \text{if $A$ doesn't occur}
\end{cases}
\]

We also talk about the expected value of a such variable. If $X_H$ is the
indicator variable that a coin flip comes up heads $I\{H\}$ then we have
$\text{Pr}\{H\} = 1/2$, thus the expected value
\[
E[X_H] = 1\cdot \text{Pr}\{H\} + 0\cdot \text{Pr}\{\ol{H}\} = 1/2
\]

In general we have $E[X_A] = \text{Pr}\{A\}$ (by definition basically).

If we want to compute how many heads we expect from a series of $n$ coin flips
we could use the variables $X_I = I\{\text{the $i$th flip is heads}\}$. We can
then sum all these $n$ variables together:

\[
X = \sum_{i=1}^n X_i \Leftrightarrow
    E[X] = E\left[ \sum_{i=1}^n X_i\right] = \sum_{i=1}^n E[X_i]
\]

because of linearity of expectation. This is just $n/2$.

\subsection{Shuffling arrays}
The book presents two ways of doing this. One is to assign each number a
priority $\proc{Random}(1,n^3)$ and sort using this priority. If we
disregard that some indices may get the same priority we can prove this by
letting $E_i$ be the event that $A[i]$ gets the $i$th smallest priority. Then
the probability that all $E_i$ occurs are:

\[\text{Pr}\{E_1\cap E_2\cap \ldots \cap E_n\}\]

This is the same as
\[\text{Pr}\{E_1\}\cdot \text{Pr}\{E_2 | E_1\} \cdot
    \text{Pr}\{E_3 | E_2\cap E_1\} \ldots
\]

That is. The chance that $E_2$ happens IF $E_1$ has already happened. Thus
The chance that $A[2]$ is the lowest of the remaining $n-1$ numbers. This is
just
\[1/n \cdot 1/(n-1) \cdot \ldots \cdot 1/2 \cdot 1 = 1/n!\]
Thus proving the shuffling.
\\\\
The other way of shuffling is the Knuth Shuffle. We go through the array and
at each point we pick $A[i]$ at random from the array $A[i..n]$. To prove
this we use the following invariant:

\textit{Prior to the $i$th iteration each possible $(i-1)$-permutation has
probability $(n-i+1)!/n!$ of being the one in the array $A[1..i-1]$.}

\begin{description}
\item [Initialization:] The array $A[1..0]$ is empty and has probability
    $(n-i+1)!/n! = n!/n! = 1$ of being the only $0$-permutation.
\item [Maintenance:] Consider any $i$-permutation. If we know that each
    $i-1$-permutation has probability $(n-i+1)!/n!$ of being in $A[1..i-1]$
    and let's call this event $E_1$ for the specific $i$-permutation's first
    $i-1$ numbers. We then have probability $n-i+1$ of picking the last element
    of the permutation to position $A[i]$. Thus we get:
    \[\text{Pr}\{E_2\cap E_1\} = \text{Pr}\{E_2 | E_1\}\text{Pr}\{E_1\}\]
    is equal to
    \[\frac{1}{n-i+1}\frac{(n-i+1)!}{n!} = \frac{(n-i)!}{n!}\]
\item [Termination:] At termination $i = n+1$ thus
    \[\frac{(n-(n+1) + 1}{n!} = \frac{0!}{n!} = \frac{1}{n!}\]
\end{description}


\subsection{Examples}
\subsubsection{Hiring problem}
We want to hire some people for a project. Each person has a ``quality'', so
we know which are best. It costs $c_h$ to hire people and the tactic we use
is to always hire a candidate if he is better than the previously hired one.
So if the candidates arrive in order $(4,1,5,3,8,2,7,6)$ we will hire
$(4,5,8)$. 

Let $X_i = I\{\text{candidate $i$ is hired}\}$. Then we have
$X = X_1 + \ldots + X_n$ where $X$ is the number of candidates hired. We have
that each candidate in $1..i$ is equally likely to be the best of the first $i$
candidates. Thus we get:

\begin{align}
E[X] &= E\left[ \sum_{i=1}^n X_i\right]\\
     &= \sum_{i=1}^n E[X_i] \\
     &= \sum_{i=1}^n 1/i \\
     &= O(\ln n)
\end{align}

\textbf{Randomization:} We cannot be sure about the distribution of the input
thus we want to randomize it. Let's say the algorithm gets an array $A$ of
candidates, then we can start by shuffling this array.

\subsubsection{Quicksort}
We pick the pivot at random from the $r-p+1$ elements.

The analysis looks at how many comparisons we do. Let $X_{ij}$ be the
indicator random variable for whether $i$ and $j$ are compared. (They will
only be compared zero or one times because one of them must be the pivot).
Let $Z_{ij}$ be the sorted array $A'[i..j]$. Then $i$ and $j$ will only be
compared if one of them is the first pivot chosen in $Z_{ij}$. Thus we have:
\[E[X_{ij}] = \frac{2}{j-i+1}\]
and thus:
\begin{align}
E[X] &= \sum_{i=1}^{n-1}\sum_{j = i+1}^n \frac{2}{j-i+1} \\
     &= \sum_{i=1}^{n-1}\sum_{k = 1}^{n-i} \frac{2}{k+1} \\
     &< \sum_{i=1}^{n-1}\sum_{k=1}^n \frac{2}{k} \\
     &= O(n\lg n)
\end{align}


\subsubsection{Selection}
We use randomized-partition and do a quicksort except we only sort half the
array.. basically.

Let $X_k$ be the indicator random variable that the array $A[p..q]$ has exactly
$k$ elements. Because each pivot is equally likely to be chosen we have
$E[X_k] = 1/n$. We upper bound the selection by assuming that $i$ always falls
in the largest of the two arrays $A[1..k-1]$ and $A[k+1..n]$. Thus we have:

\begin{align}
T(n) &\le \sum_{k=1}^n X_k\cdot (T(\max(k-1,n-k)) + O(n)) \\
     &=   \sum_{k=1}^n X_k\cdot T(\max(k-1,n-k)) + O(n)
\end{align}

Taking expectations:

\begin{align}
E[T(n)] &= \sum_{k=1}^n E[X_k\cdot T(\max(k-1,n-k))] + O(n) \\
        &= \sum_{k=1}^n E[X_k] \cdot E[T(\max(k-1,n-k))] + O(n) \\
\end{align}

This is because $X_k$ and $T(\max(k-1,n-k))$ are independant. We also have
$E[X_k] = 1/n$. Because we take $\max(k-1,n-k)$ we have that each term from
$\lfloor n/2 \rfloor .. n-1$ appears twice:

\[
E[T(n)] \le \frac{2}{n} \sum_{k=\lfloor n/2 \rfloor}^{n-1} E[T(k)] + O(n)
\]

This can be show n by substitution to be $O(n)$. Not really relevant for this
course. It's just ``simple'' math.


\subsubsection{Binary Search Trees}
The proof that a randomly build BST has expected height $O(\lg n)$ goes as
follows:

\begin{enumerate}
\item We have a binary tree built on $n$ keys. Let $X_n$ be the height of this
    tree.
\item Let $Y_n$ be the exponential height, $Y_n = 2^{X_n}$.
\item Let $R_n$ be the rank of the root. If we built the tree on the set
    $\{1,2,\ldots, n\}$ then $R_n$ is equally likely to be any of those.
\item Let $Z_{n,i}$ be the indicator random variable that
    $\text{Pr}\{R_n = i\}$. We thus have $Z_{n,i} = 1/n$.
\item We have $Y_n = 2\cdot \max(Y_{i-1}, Y_{n-i})$. Or $Y_n$ is two times the
    exponential height of its biggest subtree.
\item $Y_{n-i}$ and $Y_{i-1}$ are independant of $Z_{n,i}$ except that it
    defines how many elements there are in this tree. $Y_{i-1}$ is just like
    any other tree with $i-1$ nodes.
\item This all results in:
    \[Y_n = \sum_{i=1}^n Z_{n,i} (2\cdot \max(Y_{i-1},Y_{n-i}))\]
    Taking expectations and using linearity and independence we get:
    \begin{align}
    E[Y_n] &=   \sum_{i=1}^n E[Z_{n,i}] E[2\cdot \max(Y_{i-1},Y_{n-i})] \\
           &=   \frac{2}{n} \sum_{i=1}^n E[\max(Y_{i-1},Y_{n-i})] \\
           &\le \frac{2}{n} \sum_{i=1}^n (E[Y_{i-1}] + E[Y_{n-i}]) \\
           &=   \frac{4}{n} \sum_{i=0}^{n-1} E[Y_i]
    \end{align}
    Line two we move the $2$ out of the second $E[..]$. Line three follows from
    some excercise. Line three is because each term appears twice.
\end{enumerate}

If we use the substitution method with the guess:
\[E[Y_n] \le \frac{1}{4} \begin{pmatrix}n+3 \\ 3\end{pmatrix}\]
and the identity:
\[\sum_{i=0}^{n-1} \begin{pmatrix}i+3 \\ 3\end{pmatrix} =
    \begin{pmatrix}n+3 \\ 4 \end{pmatrix}
\]
We can see that $Y_n$ is polynomial in $n$ and thus $E[X_n]$ is $O(\lg n)$.

\subsubsection{3-CNF}
We randomly set each literal to $1$ with probability $1/2$ and $0$ with same
probability. This version of 3-CNF requires exactly $3$ literals per clause.
No literal and its negation in same clause. No literal more than once per
clause.

Let $Y_i = I\{\text{clause $i$ is satisfied}\}$

A clause is only satisfied when all its literals are set to $0$. This means
$E[Y_i] = 7/8$. Summing all $Y_i$s up gives $7m/8$ and thus a
randomized $8/7$-approximation.

\end{document}
