\documentclass[a4paper]{article}

%Packages
\usepackage[english]{babel}
\usepackage[ansinew]{inputenc}
\usepackage{amsmath, amssymb, amsthm}
\usepackage{cancel}
\usepackage[T1]{fontenc}
\usepackage{graphicx}
\usepackage[stable]{footmisc}
\usepackage{lastpage}
\usepackage{ulem}
\usepackage{listings}
\usepackage[usenames,dvipsnames]{color}
\usepackage{clrscode3e}
\usepackage{multirow}
\usepackage{hyperref}

%Listings
\definecolor{listingShade}{RGB}{245,245,245}
\lstset{ %
%language=PHP, %LANGUAGE HERE! :D
basicstyle=\ttfamily,
numbers=left,
numberstyle=\ttfamily\footnotesize,
stepnumber=1,
numbersep=5pt,
backgroundcolor=\color{listingShade},
showspaces=false,
showstringspaces=false,
showtabs=false,
frame=tb,
tabsize=8,
captionpos=b,
breaklines=true,
breakatwhitespace=false,
title=\lstname
}

%Theorems
\theoremstyle{definition}
\newtheorem{thm}{Theorem}[section]
\theoremstyle{definition}
\newtheorem{lem}{Lemma}[section]
%\theoremstyle{definition}
%\newtheorem{defi}{Definition}%[section]

%Math sets
\newcommand{\N}{\mathbb N} %the natural numbers
\newcommand{\Z}{\mathbb Z} %the integers
\newcommand{\Q}{\mathbb Q} %the rational numbers
\newcommand{\R}{\mathbb R} %the real numbers
\newcommand{\C}{\mathbb C} %the complex numbers

%Other commands
\newcommand{\ul}{\underline}
\newcommand{\ol}{\overline}
\newcommand{\mono}[1]{{\ttfamily#1}}

%Title stuff
\title{Advanced algorithms\\Convex Hull Notes}
\author{S�ren Dahlgaard}

\begin{document}

\maketitle

\section{Disposition}
\begin{enumerate}
\item Convex hull? Minimum convex set.
\item Graham's Scan + Jarvis march, general idea
\item Lower bound with sorting + intuition it's wrong.
\item Marriage before conquest
\begin{enumerate}
\item Upper/lower identical.
\item General procedure: median, bridge, recurse.
\item Finding the bridge. Intuition of slope and removing
\item Time complexity
\end{enumerate}
\item Chan's algorithm
\begin{enumerate}
\item Idea. Merge graham and jarvis.
\item Make incremental guesses at $h$ and stop if we're wrong.
\item Time complexity.
\end{enumerate}
\end{enumerate}

\section{Convex hull}
Given a set $P$ of points in the plane $E^2$ (or space $E^3$ and more, but we
will only look at the 2D example) we want to find the smallest convex set
$C\subseteq P$ such that every point in $P$ is in $C$. We can see it as having
a board with nails and wrapping a rubber band around it.

A convex set is a set such that if we pick any point $a,b$ in it, then the line
from $a$ to $b$ also lies entirely within the set. Or more exactly. For any
$t$ in the interval $[0..1]$ we must have
\[(1-t)x + ty\]
is in the set.

We require that $|P|\ge 3$. For most of the algorithm we will also require that
no three points are colinear and that no points have the same $x$ or $y$
coordinates. These are reasonable assumptions for the continuous world. They
just add a lot of special cases which is normal with computational geometry.

\subsection{Graham's scan}
The idea is to start with a point $p_0$ that has to be in $\text{CH}(P)$.
We then iterate through the points of $P$ in order of their polar angle
with respect to $p_0$. We know that $p_1$ and $p_m$ must also be in $C$.
We keep the current ``convex hull'' in a stack $S$ and for each $p_i$ we do:

\begin{enumerate}
\item While $p_i$ does not make a left turn with the two top poins of $S$ we
    pop the top element.
\item Add $p_i$ to $S$.
\end{enumerate}

This clearly runs in $O(n\lg n)$ because we sort the points and then each point
is at most pushed and popped once.

Correctness can be shown with the following invariant:

\textit{At the start of the $i$th iteration, $S$ contains
$\text{CH}(\{p_0,p_1,\ldots,p_{i-1}\})$ in sorted order.}

\subsection{Jarvis March}
Jarvis's March uses ``gift wrapping''. Again we pick a point that has to be in
the hull (the bottom-most, left-most point). We then act like taking some gift
paper and stretch it all the way to the right. We then start going up until it
touches another point. In other words, we pick the point $p_i$ that maximizes
the angle $\angle p_{i-2}p_{i-1}p_i$. This can easily be implemented in
$O(nh)$.

To optimize this we might split the run into two. We continue wrapping until we
reach $p_k$ with the maximum $y$-value. Then we have constructed the
right-chain (assuming no identical $y$-values). Do the opposite for the
left-chain.


\subsection{Lower bound}
Some people suggested that $n\lg n$ might be a lower bound because we could
otherwise use it to sort $n$ numbers by finding the convex hull of
$(a_i, a_i^2)$ for all these $n$ points. While this is true, we can still
make algorithms in time $O(nh)$ like Jarvis' March, where $h$ can be less
that $\lg n$, surely. Note that in the sorting case we have $h=n$, therefore
we might believe that it can be done in $O(n\lg h)$, which we provide to
algorithms for below.


\subsection{Marriage-before-conquest}
Marriage-Before-Conquest is actually a modification of divide-and-conquer.
Before solving the subproblems recursively we look at how they will combine
in the end and remove a bunch of the problem (in this case points) based on
this.

The idea is to compute the top of the hull and the bottom of the hull
seperately. We can combine this in constant time adding at most two vertical
edges (if the left/rightmost points have different $y$-values).

The algorithm runs as follows:

\begin{enumerate}
\item Find the median $x$-value of the points and create the line
    $L = \{(x,y) : x = m\}$, where $m$ is the median $x$-value.
\item Find the part, $b$, of the top hull crossing this line. (The bridge).
\item Delete all points under $b$.
\item Recursively solve $P_l$ and $P_r$ the points left and right of the
    bridge.
\end{enumerate}

The hard part here is to find the bridge. We use the following idea: A
supporting line is a non-vertical straight line that contains at least one
point of $P$ such that no other points of $P$ are above the line.

We use a couple of lemmas:

\textit{Let $p,q\in P$. If $x(p) = x(q)$ and $y(p) < y(q)$ then obviously
$p$ cannot be a bridge point.}

\textit{Let $x(p) < x(q)$. If the slope $s_{pq} > s_b$ then $p$ cannot be a   
bridge point.} proof: Then $q$ would be above the bridge supporting line,     
which is a contradiction. The reverse case $s_{pq} < s_b$ implies that $q$    
isn't a bridge point.                                                         

\textit{Let $h$ be a supporting line. $s_h < s_b$ iff $h$ only contains points
to the right of $L$. $s_h = s_b$ iff $h$ has a point on either side.}

We now find the bridge like so:
\begin{enumerate}
\item Split $P$ into $|P|/2$ pairs of points.
\item Find the median slope of the $|P|/2$ line segments.
\item Translate the slope to make a supporting line (can be done in $O(|P|)$).
\item If this line $h$ only contains points right of $L$ we have $s_h < s_b$
    and thus all of the $|P|/2$ lines $pq$ with slope $s_{pq} < s_h < s_b$ can
    not have $q$ as bridge points. This will eliminat $|P|/4$ points because we
    chose the median slope.
\item Do this recursively.
\end{enumerate}

The running time of the bridge function is:
\[T(n) = T(3n/4) + O(n) = O(n)\]
This gives a total running time of:
\[
T(n,h) = \max_{h_l+h_r = h} \{ T(n/2, h_l) + T(n/2, h_r)\} + O(n)
\]
Because we have at most $n/$ points to the left of the median and $n/2$ points
to the right. The base case here is $T(n,h) = O(n)$ when $h = 2$. This gives us
that the worst case of the above recursion is when $h_l = h_r = h/2$. We can
show that $T(n,h) = n\log h$ using the substitution method:

\begin{align}
T(n,h) &\le \max_{h_l + h_r = h}
    \left\{c\frac{n}{2}\log h_l + c\frac{n}{2}\log h_r\right\} + O(n) \\
       &=   \max_{h_l+h_r=h}\{\log(h_lh_r)\} + O(n) + c\frac{n}{2} \\
       &\le O(n) + c\frac{n}{2} \log(h/2)^2 \\
       &=   n\log h
\end{align}


\subsection{Chan's algorithm}
The big problem about the marriage-before-conquest algorithm is that
it relies on finding the median in linear time. This is very costly in
practice thus we strive to find a better algorithm.\\\\

The idea of Chan's algorithm is to preprocess the points and splitting them
into $n/m$ convex hulls which we compute with graham's scan in
$n/m \cdot m\lg m = n\lg m$. If $m=h$ this is obviously $n\lg h$.

With this preprocessing we can do a modified version of jarvis' march.
Finding the point $p_i$ that maximizes $\angle p_{i-2}p_{i-1}p_i$ can now
be done in $O(n/m\lg m)$ by finding the point for each of the convex hull
with binary search and taking the best among these. This gives a total
running time of:

\[
T(n) = O(n/m\cdot m\lg m + h((n/m)\lg m)) = O(n\lg m + h((n/m)\lg m) =
    O(n(1 + h/m)\lg m)
\]

Now, if we pick $m = h$ this will be $O(n\lg h)$ like the mbc algorithm. If
we pick $m$ too large it will approach $n(1 + 0)\lg n$, and if we pick it
too low it will approach $O(n(1+h)\lg 1)$. The idea is thus to make a
increasing guesses at $h$ (our guess is $H$) and $m$ (which we want to be $h$).
If the algorithm doesn't calculate a hull in $H$ steps we have guessed too low
and we stop it before it runs for too long. Thus we make guesses:

\[(2^{2^1}, 2^{2^2}, 2^{2^3}, \ldots)\]
until $2^{2^t} \ge h$. We set $m=H=2^{2^t}$. Thus the $t$th iteration takes
\[
T(n) = O(n(1 + H/H)\lg H) = O(n/lg H) = O(n2^t)
\]
Thus the total running time is:
\begin{align}
T(n) &= \sum_{t=1}^{\lceil \lg \lg h\rceil} O(n2^t) \\
     &= O(n2^{\lceil \lg \lg h \rceil + 1}) \\
     &= O(n\lg h)
\end{align}

We might improve this algorithm by picking a better $m$ or $H$ value, removing
the points found to be inside $C_i$ when we run graham's scan or more ways.
\end{document}
