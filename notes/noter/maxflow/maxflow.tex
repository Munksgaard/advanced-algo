\documentclass[a4paper]{article}

%Packages
\usepackage[english]{babel}
\usepackage[ansinew]{inputenc}
\usepackage{amsmath, amssymb, amsthm}
\usepackage{cancel}
\usepackage[T1]{fontenc}
\usepackage{graphicx}
\usepackage[stable]{footmisc}
\usepackage{lastpage}
\usepackage{ulem}
\usepackage{listings}
\usepackage[usenames,dvipsnames]{color}
\usepackage{clrscode3e}
\usepackage{multirow}
\usepackage{hyperref}

%Listings
\definecolor{listingShade}{RGB}{245,245,245}
\lstset{ %
%language=PHP, %LANGUAGE HERE! :D
basicstyle=\ttfamily,
numbers=left,
numberstyle=\ttfamily\footnotesize,
stepnumber=1,
numbersep=5pt,
backgroundcolor=\color{listingShade},
showspaces=false,
showstringspaces=false,
showtabs=false,
frame=tb,
tabsize=8,
captionpos=b,
breaklines=true,
breakatwhitespace=false,
title=\lstname
}

%Theorems
\theoremstyle{definition}
\newtheorem{thm}{Theorem}[section]
\theoremstyle{definition}
\newtheorem{lem}{Lemma}[section]
%\theoremstyle{definition}
%\newtheorem{defi}{Definition}%[section]

%Math sets
\newcommand{\N}{\mathbb N} %the natural numbers
\newcommand{\Z}{\mathbb Z} %the integers
\newcommand{\Q}{\mathbb Q} %the rational numbers
\newcommand{\R}{\mathbb R} %the real numbers
\newcommand{\C}{\mathbb C} %the complex numbers

%Other commands
\newcommand{\ul}{\underline}
\newcommand{\mono}[1]{{\ttfamily#1}}

%Title stuff
\title{Advanced algorithms\\Maximum flow noter}
\author{S�ren Dahlgaard}

\begin{document}

\maketitle

\section{Disposition}
\begin{enumerate}
\item Flow network.
\item Flows in flow networks - cap constraint, flow conservation.
\item Antiparallel edges / multiple sinks/sources
\item Residual networks
\item Augmenting paths - they are flows in $R_f$.
\item $(f\uparrow f')$ is a flow in $G$ - don't prove
\item Cuts - duality. $f(S, V\setminus S) = |f|$ for all cuts.
\item $|f| \le c(S,V\setminus S)$
\item Duality. maxflow $\Rightarrow$ no augment $\Rightarrow$ mincut
    $\Rightarrow$ maxflow.
\item Bad time
\item Edmonds-Karp. Edge can be critical at most $|V|/2$ times.
\item Maximum bipartite matching. Define and show similarity.
\end{enumerate}



\section{Summary}
Vi skal nu kigge p� et super interessant problem, Maximum Flow. Det           
er et problem jeg kiggede p� f�rste gang i 1. g efter at have m�dt            
f�lgende problem: \url{http://www.boi2007.de/tasks/escape.pdf} -              
Det er nok ingen hemmelighed at l�sningen best�r i at bruge maximum           
flow. Ved mange andre konkurrencer er vi blevet stillet problemer             
der kunne l�ses med max-flow (se f.eks. F - Risk fra dette �rs NWERC          
\url{http://2010.nwerc.eu/results/nwerc2010-problemset.pdf})


\section{Flow netv�rk}

For at kunne finde det maksimale flow er vi n�dt til at definere hvad et      
flow er.Lad $G = (V,E)$ v�re en graf og lad der v�re en kapacitetsfunktion   
$c(u,v)\ge 0$ for alle kanter $(u,v)$. For alle $(u,v)\ne E$ definerer vi     
$c(u,v)=0$. Vi har ogs� to s�rlige knuder i grafen $s,t$ hhv. source og sink. 
Vi antager ogs�, at alle knuder ligger p� en sti mellem $s$ og $t$. Yderligere
kr�ver vi, at hvis $(u,v)\in E$ s� er $(v,u)\notin E$

 For hver kant definerer vi yderligere et "flow" $f : V\times V\to            
\mathbb{R}$. Dette flow skal overholde f�lgende to krav:                      

\begin{description}
\item [Capacity constraint] For alle $u,v\in V$ skal det g�lde at
     $0\le f(u,v)\le c(u,v)$
\item [Flow conservation] For alle $u\in V\setminus \{s,t\}$ skal det g�lde,
    at:
    \[\sum_{v\in V} f(v,u) = \sum_{v\in V} f(u,v)\]
\end{description}

Vi definerer v�rdien af et flow, $|f|$ som overskudet der efterlade vores
source (se herunder). Alternativt kunne vi definere det som overskudet af flow
der l�ber ind i vores sink.

\[\sum_{v\in V} f(s,v) - \sum_{v\in V} f(v,s)\]


\subsection{At transformere diverse tilf�lde til flow netv�rk}

\begin{description}
\item [Antiparallele kanter]
Et meget normalt tilf�lde er at der er kanter i begge retninger mellem knuder.
Dvs. vi har b�de $(u,v), (v,u)\in E$. I dette tilf�lde erstatter vi den ene
kant $(u,v)$ med to kanter $(u,w)$ og $(w,v)$. Vi s�tter
$c(u,w) = c(w,v) = c(u,v)$.
\item [Super source/sink]
Hvis der er mere end en source eller sink laver vi en ny ``super source'' hhv.
sink. Vi forbinder super source til alle sources og lader kapaciteten mellem
disse v�re $\infty$.
\end{description}

TODO: add proofs.

\section{Ford-Fulkerson method}

General algorithm:

\begin{codebox}
\Procname{$\proc{Ford-Fulkerson}(G, s, t)$}
\li initialize flow $f$ to 0
\li \While there exists augmenting path from $s$ to $t$ in residual network
\li \Do
        Augment flow along the path
    \End
\li \Return $f$
\end{codebox}

\subsection{Residual network}
For a graph $G$ and flow $f$ we define a residual network $G_f$, where the
capacity of an edge is $c_f(u,v) = c(u,v) - f(u,v)$. We also define an edge
in the opposite direction with capacity $c(v,u) = f(u,v)$. Now it is nice that
we cannot have both $(u,v)$ and $(v,u)$ in $E$.

It is clear that $|E_f| = O(|E|)$. Or more exactly $|E_f| \le 2|E|$.

Other than the fact that $G_f$ can contain both edges $(u,v)$ and $(v,u)$ we
can talk about it as a flow network with respect to $c_f$.

We talk about flow augmentation. Define $f\uparrow f'$ as the augmentation of
$f$ by $f'$. Define it as:

\[
(f\uparrow f') =
\begin{cases}
f(u,v) + f'(u,v) - f'(v,u) & (u,v)\in E \\
0 & \text{otherwise}
\end{cases}
\]

We claim that $f\uparrow f'$ is a flow in $G$ with
$|f\uparrow f'| = |f| + |f'|$.

\begin{proof}
First show that the capacity constraint is satisfied:

\begin{align}
(f\uparrow f')(u,v) &=   f(u,v) + f'(u,v) - f'(v,u) \\
                    &\ge f(u,v) + f'(u,v) - f(u,v)  \\
                    &=   f'(u,v)                    \\
                    &\ge 0
\end{align}
and
\begin{align}
(f\uparrow f')(u,v) &=   f(u,v) + f'(u,v) - f'(v,u) \\
                    &\le f(u,v) + f'(u,v)           \\
                    &\le f(u,v) + c_f(u,v)          \\
                    &=   f(u,v) + c(u,v) - f(u,v)   \\
                    &=   c(u,v)
\end{align}

Flow conservation:

\begin{align}
\sum_{v\in V}(f\uparrow f')(u,v) &=
    \sum_{v\in V}(f(u,v) + f'(u,v) - f'(v,u)) \\
&=  \sum_{v\in V}f(u,v) + \sum_{v\in V}f'(u,v) - \sum_{v\in V}f'(v,u) \\
&=  \sum_{v\in V}f(v,u) + \sum_{v\in V}f'(v,u) - \sum_{v\in V}f'(u,v) \\
&=  \sum_{v\in V}(f(v,u) + f'(v,u) - f'(u,v)) \\
&=  \sum_{v\in V}(f\uparrow f')(v,u)
\end{align}

Third line follows from flow conservation. Ask someone about the last two
summations being equal. This shows that it is a flow in $G$. We now compute
the value.

For any vertex $v\in V$ we can have either $(s,v)$ or $(v,s)$ but never both.
Define two sets: $V_1 = \{v : (s,v)\in E\}$ and $V_2 = \{v : (v,s)\in E\}$.
Clearly $V_1\cup V_2 \subseteq V$ and $V_1\cap V_2 = \emptyset$. We have

\begin{align}
|f\uparrow f'| &=
   \sum_{v\in V}(f\uparrow f')(s,v) - \sum_{v\in V}(f\uparrow f')(v,s) \\
&= \sum_{v\in V_1}(f\uparrow f')(s,v) - \sum_{v\in V_2}(f\uparrow f')(v,s) \\
&= \sum_{v\in V_1}(f(s,v) + f'(s,v) - f'(v,s)) -
        \sum_{v\in V_2}(f(v,s) + f'(v,s) - f'(s,v)) \\
&= \sum_{v\in V_1}f(s,v) + \sum_{v\in V_1}f'(s,v) - \sum_{v\in V_1}f'(v,s) -
    \sum_{v\in V_2}f(v,s) - \sum_{v\in V_2}f'(v,s) + \sum_{v\in V_2}f'(s,v) \\
&= \sum_{v\in V_1}f(s,v) - \sum_{v\in V_2}f(v,s) +
    \sum_{v\in V_1\cup V_2}f'(s,v) - \sum_{v\in V_1\cup V_2}f')v,s)
\end{align}

and so on...
\end{proof}

\subsection{Augmenting paths}
Augmenting path is a simple path, $p$, from $s$ to $t$ in $G_f$.

The capacity of an augmenting path is the of a critical edge on $p$. We can
write this as $c_f(p) = \min\{c_f(u,v) : (u,v)\in p\}$.

We can define a flow in $G_f$ from a path $p$ as:

\[
f_p(u,v) =
\begin{cases}
c_f(p) & \text{if } (u,v)\in p \\
0 & \text{otherwise}
\end{cases}
\]

It is now obvious that for any augmenting path $p$ we have $f\uparrow f_p$ as
a flow in $G$ with $|f\uparrow f_p| = |f| + |f_p| > |f|$.


\subsection{Cuts of flow networks}
A cut is almost the classic definition except we require that $s\in S$ and
$t\in V\setminus S = T$. We define both the flow across a cut and the
capacity of a cut:

\[
f(S,T) = \sum_{u\in S}\sum_{v\in T}f(u,v) - \sum_{u\in S}\sum_{v\in T}f(v,u)
\]

\[
c(S,T) = \sum_{u\in S}\sum_{v\in T}c(u,v)
\]

A minimum cut is a cut with minimal capacity of all cuts.

For any cut $(S,T)$ we have $f(S,T) = |f|$

\begin{proof}
First we look at the flow definition:
\begin{align}
|f| &= \sum_{v\in V}f(s,v) - \sum_{v\in V}f(v,s) \\
    &= \sum_{v\in V}f(s,v) - \sum_{v\in V}f(v,s) +
    \sum_{u\in S\setminus\{s\}}\left(\sum_{v\in V}f(u,v) -
                                     \sum_{v\in V}f(v,u)\right)
\end{align}
See p. 722. Main ideas:
\begin{enumerate}
\item Expand the rightmost summation
\item Regroup into summations over $v\in V$ with edges going into $v$. And
    another group with edges going out of $v$.
\item Split all $v\in V$ into $v\in S$ and $v\in T$. Uses $S\cup T = V$.
\item Cancel out terms. and we get what we want.
\end{enumerate}
\end{proof}

This means that $|f|$ is bounded above by the capacity of ALL cuts.
Especially the flow is bounded above by the capacity of a minimum cut.

Max-flow Min-cut \textit{duality} (pawel likes this word) is that the following
are equivalent:

\begin{enumerate}
\item $f$ is a maximum flow in $G$
\item There is no augmenting path in $G_f$
\item $|f| = c(S,T)$ for some cut $(S,T)$ in $G$.
\end{enumerate}

\begin{proof}
$1\Rightarrow 2$: If there was an augmenting path $f_p$, then $f\uparrow f_p$
would be a flow in $G$ with bigger value. Contradiction

$2\Rightarrow 3$: Look at the cut $S = \{v\in V : \text{ There exists a path
from $s$ to $v$ in $G_f$}\}$. Look at any $u\in S$ and $v\in T$. if
$(u,v)\in E$ we must have $f(u,v) = c(u,v)$ (otherwise $v\in S$). Conversely
if $(v,u)\in E$ we must have $f(v,u) = 0$ or $c_f(u,v) = f(v,u)$ would mean
$v\in S$. Now check that $f(S,T) = c(S,T)$.

$3\Rightarrow 1$: We cannot have $|f| > c(S,T)$.
\end{proof}

\subsection{Analysis}
If capacities are irrational we have $T(n) = O(\infty)$. If capacities are
rational we can scale them to be integers. If capacities are integers we have
$T(n) = O(E |f^*|$. The while loop is executed at most $|f^*|$ times because
we always get a strictly better flow. The stuff in the while loop can be done
in linear time.

\section{Edmonds-Karp}
Using a BFS for ford-fulkerson gives $O(VE^2)$ running time.

First step on proving this is to see that the minimum path length (amount of
edges) in $G_f$ increases monotonically. Let $\delta_f(s,v)$ be the minimum path
length from $s$ to $v$ in $G_f$.

\begin{proof}
Look at the smallest $\delta_{f'}(s,v)$ that changed when augmenting $f$ with
$f'$. Let $u$ be the vertex before $v$ in the path from $s$ to $v$. We must
have $\delta_{f'}(s,u) = \delta_{f'}(s,v) - 1$. We also have
$\delta_f(s,u) = \delta_{f'}(s,u)$. Therefore $(u,v)\ne E_f$.

We must therefore have had $\delta_f(s,v) = \delta_f(s,u) - 1$ which leads
to contradiction.
\end{proof}

We can use this to show that an edge can at most be critical $|V|/2$ times.

Simply look at $\delta_f(s,v) = \delta_f(s,u) + 1$ when $(u,v)$ is critical.
In order for the edge to return to the residual network we must have $(v,u)$
be critical. This can only happen when
$\delta_{f'}(s,u) = \delta_{f'}(s,v) + 1$. Since
$\delta_{f'}(s,v) + 1 \ge \delta_f(s,v) +1 = \delta_f(s,u) + 2$ the result
follows.

Each augmenting path has at least one critical edge and only $O(VE)$ times can
there be a critical edge. This gives the running time.


\section{Maximum bipartite matching}

A matching is a subset $M\subseteq E$ such that for all vertices $v\in V$ at
most one edge in $M$ is incident to $v$. A maximum matching is a matching $M$
that has $|M| \ge |M'|$ for any other matching $M'$.

We can create a graph $G'$ with nodes $V' = V \cup \{s,t\}$ and edges

\[E' = \{(s,u) : u\in L\}\cup \{(u,v) : (u,v)\in E\} \cup \{(u,t) : u\in R\}\]

All capacities $c(u,v) = 1$. It is clear that $|E'| = O(|E|)$ because we
assume each node $v\in L\cup R$ has one edge incident in $E$.

We wanna show that a flow $f$ in $G'$ corresponds to a matching $M$ in $G$.
First we say that a flow is integer-valued if $f(u,v)$ is integer for all
$(u,v)\in V\times V$. The claim is that if $M$ is a matching in $G$ it
corresponds to a flow $f$ in $G'$ with $|f| = |M|$ and the other way around.

\begin{proof}
For a matching $M$ create a flow $f(u,v) = f(s,u) = f(u,t) = 1$ for all
$(u,v)\in M$. It is easy to see that the constraints are satisfied and that
$|f| = |M|$.

For a flow $f$ create a matching
\[M = \{(u,v) : u\in L, v\in R, \text{ and } f(u,v) > 0\]
Because $f$ is integer-valued this is okay. To see that this is a matching
use that $(s,u) = 1$ and flow conservation means that the sum over $(u,v) = 1$
(or $0$). We can also use this to show that $|M| = |f|$.
\end{proof}

We also need to show that when $c(u,v)$ is integral for all $(u,v)$ the
maximum flow found by Ford-Fulkerson will be integer-valued. This is done
easily by induction over the iteration. Base case is trivial cause $|f| = 0$.

We can now proof by contradiction that a maximum flow corresponds to a
maximum matching.
\end{document}
