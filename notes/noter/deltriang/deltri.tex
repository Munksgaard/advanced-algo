\documentclass[a4paper]{article}

%Packages
\usepackage[english]{babel}
\usepackage[ansinew]{inputenc}
\usepackage{amsmath, amssymb, amsthm}
\usepackage{cancel}
\usepackage[T1]{fontenc}
\usepackage{graphicx}
\usepackage[stable]{footmisc}
\usepackage{lastpage}
\usepackage{ulem}
\usepackage{listings}
\usepackage[usenames,dvipsnames]{color}
\usepackage{clrscode3e}
\usepackage{multirow}
\usepackage{hyperref}

%Listings
\definecolor{listingShade}{RGB}{245,245,245}
\lstset{ %
%language=PHP, %LANGUAGE HERE! :D
basicstyle=\ttfamily,
numbers=left,
numberstyle=\ttfamily\footnotesize,
stepnumber=1,
numbersep=5pt,
backgroundcolor=\color{listingShade},
showspaces=false,
showstringspaces=false,
showtabs=false,
frame=tb,
tabsize=8,
captionpos=b,
breaklines=true,
breakatwhitespace=false,
title=\lstname
}

%Theorems
\theoremstyle{definition}
\newtheorem{thm}{Theorem}[section]
\theoremstyle{definition}
\newtheorem{lem}{Lemma}[section]
%\theoremstyle{definition}
%\newtheorem{defi}{Definition}%[section]

%Math sets
\newcommand{\N}{\mathbb N} %the natural numbers
\newcommand{\Z}{\mathbb Z} %the integers
\newcommand{\Q}{\mathbb Q} %the rational numbers
\newcommand{\R}{\mathbb R} %the real numbers
\newcommand{\C}{\mathbb C} %the complex numbers

%Other commands
\newcommand{\ul}{\underline}
\newcommand{\ol}{\overline}
\newcommand{\mono}[1]{{\ttfamily#1}}

%Title stuff
\title{Advanced algorithms\\Delaunay Triangulation Notes}
\author{S�ren Dahlgaard}

\begin{document}

\maketitle

\subsection{Disposition}

\begin{enumerate}
\item Motivation - Terrains, maximize minimum angle
\item Thales's Theorem - Flipping edges and legal edges
\item Voronoi diagrams and the duality
\item Delaunay graphs are plane. (use disc theorem.)
\item Correctness of Delaunay triangulation.
\item Incremental method.
\item Finding the triangle containing $p_r$
\item $p_{-1}$ and $p_{-2}$.
\item Running time.
\end{enumerate}

\subsection{Summary}

When modeling terrains we often want to express a 3D space by a 2D surface.
If we have $n$ points in the continous space we would also like to approximate
the height of the intermediate points. Essentially we want to create a function
$f : A\subset \mathbb{R}^2 \to \mathbb{R}$ that gives a point a height.
\\\\
We could simply assign a point $(x,y)$ the height of the closest point
$p\in A$. This would however result in a ugly discrete terrain. The idea is
therefore to divide the terrain into triangles where each triangle vertices
are in $A$. When we have done this we raise each point in $A$ to its height.

We also want to create a good such triangulation. Our intuition (see fig 9.3)
in the material tells us that a good triangulation is one that maximizes the
minimum angle. Thus given a triangulation of $m$ triangles we define its
angle-vector as $(\alpha_1, \alpha_2, \ldots, \alpha_{3m})$ the sorted list
of the $3m$ angles in the triangulation. Order such vectors lexicografically:
\[
(\alpha_1,\ldots,\alpha_{3m}) < (\alpha'_1,\ldots,\alpha'_{3m})
    \Leftrightarrow \exists i : \alpha_j = \alpha'_j, j<i\land
    \alpha_i < \alpha'_i
\]

Euler's
characteristic\footnote{\url{http://en.wikipedia.org/wiki/Euler_characteristic}}
gives us that there are $2n-2-k$ triangles and $3n-3-k$ edges, where $k$ is
the amount of vertices on the convex hull. This is because there is $m+1$ faces
(1 unbounded) and $(3m+k)/2$ edges - 3 edges per triangle plus $k$ of the
unbounded. Furthermore all edges are incident to two triangles). Thus
\[
2 = n - ((3m+k)/2) + (m+1)
\]

We use Thales's Theorem to prove most of the things for delaunay
triangulations. It states that if we have a circle $C$ and two points $a,b$
on the circle. Let $l$ be the line passing through $a,b$. Let $p,q$ be two
points on the circle on the same side of $l$, then $\angle apb = \angle aqb$.
If $r$ and $s$ are two points on the same side of $l$ as $p,q$ with $r$
inside $C$ and $s$ outside $C$, then:
\[
\angle arb > \angle apb = \angle aqb > \angle asb
\]

If we have an edge $\ol{p_ip_j}$ that is not on the convex hull, then we
have that it is incident to two triangles $p_ip_jp_k$ and $p_ip_jp_l$. We
call flipping edges $\ol{p_ip_j}$ with $\ol{p_kp_l}$ for an edge flip. It
changes maximum $6$ angles. We call $\ol{p_ip_j}$ illegal if:
\[
\min_{1\le i \le 6} \alpha_i < \min_{1\le i \le 6}\alpha'_i
\]

It is obvious that if we have an illegal edge in a triangulation, we can
make a better triangulation by flipping it.

Rather than checking all three angles when comparing an edge to its flipped
version we can use Thales's Theorem:

\textit{If $\ol{p_ip_j}$ is an edge incident to triangles with $p_k, p_l$.
Let $C$ be the circle with $p_i,p_j,p_k$ on its circumference. Then
$\ol{p_ip_j}$ is illegal iff $p_l$ lies within $C$.}

Thus we can find an optimal triangulation by starting with any triangulation
and flipping edges until no edges are illegal. Note that this algorithm
terminates because we improve the triangulation at each step and there is a
finite amount of triangulations.

\section{Delaunay triangulation}
A voronoi diagram is a where each point $p$ has its own face such that for
all points in the face, $p$ is the closest point.

Let us define a graph where we connect each such point $p$ with each of the
points it shares an edge. Such a graph is called a Delaunay graph. We claim
that it is planar (no edges cross except at endpoints). We prove this by
contradiction.

Firstly: \textit{The edge $\ol{p_ip_j}$ is in the delaunay graph iff there is
a circle $C_{ij}$ with $p_i, p_j$ on its boundary and no other point contained
in it. The center of this disc will lie on the voronoi edge between $p_i,p_j$}

We know this to be true and will use it: Let $t_{ij}$ be the triangle with
$p_i,p_j$ and the center of $C_{ij}$. Assum another edge $\ol{p_kp_l}$
intersects with $\ol{p_ip_j}$, then both of these points must lie outside
$C_{ij}$, and thus the edge must cross one of the edges of $t_{ij}$ that
goes from the center of $C_{ij}$. The same goes the other way, but then they
can't be disjoint voronoi cells because the edge from $p_i$ to center of
$C_{ij}$ lies entirely in $\text{Vor}(p_i)$.
\\\\
We can extend this:
\begin{enumerate}
\item points $p_i,p_j,p_k\in P$ are vertices of the same face in the delaunay
    graph of $P$ iff the circle through the points contains no other points in
    $P$.
\item Points $p_i,p_j$ form an edge in the delaunay graph iff there is a closed
    disc containing $p_i,p_j$ on its boundary such that no $p\in P$ is in this
    disc.
\end{enumerate}

This implies:

\textit{$T$ is a delaunay triangulation of $P$ iff the circumcircle of any
triangle of $T$ contains no point of $P$ in its interior.}

We have that \textit{A triangulation $T$ of $P$ is legal iff it is a
delaunay triangulation.} Proof: It is easy to see that any delaunay
triangulation is legal.

Assume for contradiction that $T$ is legal, but there is a triangle $\angle   
p_i,p_j,p_k$ such that $p_l$ is in their circumcircle. If The triangle        
$\angle p_i,p_j,p_l$ is incident to $\angle p_ip_jp_k$ then the triangulation 
is not legal and we are done. Otherwise look at $\angle p_ip_jp_m$ the        
triangle incident. We assume without loss of generality that it is the edge   
$\ol{p_ip_j}$ that separates $p_l$ from $p_k$. We also assume that
$(p_ip_jp_k, p_l)$ is the pair maximizing $\angle p_ip_lp_j$. Then we have
an illegal edge $\ol{p_jp_m}$ that we could flip with $\ol{p_ip_l}$.
\\\\
Therefore we have that any angle-optimal terrain is a delaunay triangulation.
Also any Delaunay triangulation maximizes the minimum angle over all
triangulations.

\subsection{Computing the delaunay triangulation}
We could calculate the voronoi diagram and then create the triangulation, but
we will do it in an incremental way:

\begin{enumerate}
\item Start with a huge triangle $\angle p_{-2}p_{-1}p_0$ that contains all
    of $P$.
\item For each point $p_i\in P$ at random add it and find the triangle of $T$
    that contains $p_i$.
\item If $p_i$ lies on an edge $\ol{p_jp_k}$ add an edge from $p_i$ to the
    last vertex of both the triangles incident to $\ol{p_jp_k}$. If $p_i$
    lies in a triangle add edges between $p_i$ and each of the triangle's
    vertices.
\item The edges of the triangles we have changed might be illegal now, so
    legalize them.
\end{enumerate}

If the triangle $\angle p_ip_jp_k$ was changed by adding a new point we have
to legalize the edges: $\ol{p_ip_j}, \ol{p_ip_k}, \ol{p_jp_k}$. If we
flip these edges we have to legalize the triangles affected recursively.
Note that each flip increases the angles so it cannot loop infinitely. All
new edges are incident to the added point $p_l$.

The idea behind proving the correctness of the algorithm is to see that each
edge added is legal. This is done by seeing that the original circumcircle
contains only $p_l$ in its interior, thus the new circumcircle cannot
contain any other points.
\\\\
In order to find which triangle contains the new point $p_l$ we use a
directed acyclic graph (NOT A TREE!), such that:
\begin{enumerate}
\item The ``leaves'' are exactly the triangles of $T$
\item The internal nodes are previous triangles of $T$ that were removed.
\end{enumerate}
When we split a triangle it results in at most $3$ new triangles:
\begin{itemize}
\item Insert causes either $1\to 3$ or $2\to 4$
\item Flipping causes $2\to 2$.
\end{itemize}
Each deleted triangle gets a pointer to the new triangles created from it.
\\\\
Picking $p_{-1},p_{-2}$ and handling them. Instead of actually creating these
vertices we just create them conceptually. Conceptually we have:
\begin{enumerate}
\item Let $l_{-1}$ be a horizontal line below all points in $P$. We add
    $p_{-1}$ as a point on this line with high enough $x$-coordinate such that
    no circumcircle contains $p_{-1}$ and the clockwise ordering of $P$ around
    $p_{-1}$ is the same as their lexicographical order.
\item $p_{-2}$ is the opposite except we include $p_{-1}$ in $P$ for
    circumcircles and ordering.
\end{enumerate}
With these choices we can find out the position of $p_k$ wrt. $\ol{p_ip_j}$.
The following are equivalent:
\begin{enumerate}
\item $p_j$ lies left of $\ol{p_ip_{-1}}$
\item $p_j$ lies left of $\ol{p_{-2}p_i}$
\item $p_j$ is lexicographically larger than $p_i$.
\end{enumerate}
How do we check if an edge containing $p_{-1}$ or $p_{-2}$ is legal:
\begin{itemize}
\item An edge on $\angle p_{-2}p_{-1}p_0$ is always legal
\item if $j,k,i,l \ge 0$ no special case is needed.
\item An edge $\ol{p_ip_j}$ is legal iff $\min(k,l) < \min(i,j)$.
    At most one of $i,j$ is negative (otherwise it's handled by case 1) and at
    most one of $k,l$ is negative because we just inserted one of those. If
    $\min(k,l)$ is smallest we have that it lies outside the circle defined by
    the other $3$ points.
\end{itemize}

\subsubsection{Running time}
The expected number of triangles created by the procedure is $9n + 1$.
At each step we create $k$ edges and all of these are connected to the
newly inserted point, thus $k$ is bounded above by the degree of this vertex.
We have that the graph has at most $3(r+3)-6$ where $r+3$ is the amount of
vertices (3 initial plus $r$ iterations). Thus the expected degree is
\[
\frac{\text{total degree}}{r+3} = \frac{2(3(r+3) - 9)}{r+3} = O(\frac{6r}{r})
\]
We create at most $2(k-3) - 3=9$ new triangles in each iteration so a total
of $9n$ triangles plus the first one.
\\\\
We know that the total amount of triangles created is linear, so the bounding
factor must be the running time of the triangle locations. We do $O(n)$ of
these. The idea is that if we visit a triangle $\angle p_ip_jp_k$ during our
search for a leaf, we can attribute this to some triangle destroyed during the
same iteration as $\angle p_ip_jp_k$ (either the triangle itself or an incident
triangle.) If $K(\Delta)$ denotes the amount of points in the circumcircle of
$\delta$. Each triangle can be charged for each such point at most once
(because we search for each point once), so we have the total running time is:
\[
O(n + \sum_{\Delta} |K(\Delta)|
\]
This is claimed to be $O(n\lg n)$.
\end{document}
