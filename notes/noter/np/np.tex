\documentclass[a4paper]{article}

%Packages
\usepackage[english]{babel}
\usepackage[ansinew]{inputenc}
\usepackage{amsmath, amssymb, amsthm}
\usepackage{cancel}
\usepackage[T1]{fontenc}
\usepackage{graphicx}
\usepackage[stable]{footmisc}
\usepackage{lastpage}
\usepackage{ulem}
\usepackage{listings}
\usepackage[usenames,dvipsnames]{color}
\usepackage{clrscode3e}
\usepackage{multirow}
\usepackage{hyperref}

%Listings
\definecolor{listingShade}{RGB}{245,245,245}
\lstset{ %
%language=PHP, %LANGUAGE HERE! :D
basicstyle=\ttfamily,
numbers=left,
numberstyle=\ttfamily\footnotesize,
stepnumber=1,
numbersep=5pt,
backgroundcolor=\color{listingShade},
showspaces=false,
showstringspaces=false,
showtabs=false,
frame=tb,
tabsize=8,
captionpos=b,
breaklines=true,
breakatwhitespace=false,
title=\lstname
}

%Theorems
\theoremstyle{definition}
\newtheorem{thm}{Theorem}[section]
\theoremstyle{definition}
\newtheorem{lem}{Lemma}[section]
%\theoremstyle{definition}
%\newtheorem{defi}{Definition}%[section]

%Math sets
\newcommand{\N}{\mathbb N} %the natural numbers
\newcommand{\Z}{\mathbb Z} %the integers
\newcommand{\Q}{\mathbb Q} %the rational numbers
\newcommand{\R}{\mathbb R} %the real numbers
\newcommand{\C}{\mathbb C} %the complex numbers

%Other commands
\newcommand{\ul}{\underline}
\newcommand{\ol}{\overline}
\newcommand{\mono}[1]{{\ttfamily#1}}

%Title stuff
\title{Advanced algorithms\\NP Completeness Notes}
\author{S�ren Dahlgaard}

\begin{document}

\maketitle

\section{Disposition}
\begin{enumerate}
\item Formal notion. Relation $Q$ from $I\to S$. We want $S = \{0,1\}$
\item Decision problems - easier argumentation.
\item Encodings. Input length vs size.
\item Languages. Problem is defined by $L = \{x\in \Sigma^* : Q(x) = 1\}$.
\item Accept, decide. $P$
\item Polynomial-time verification, $NP$, $co-NP$.
\item Definition of NP-Complete. How do we show this?
\item Reductions and proofs
\item Sketch of circuit-sat
\item Clique example.
\end{enumerate}


\section{Formal notion}

We first want a formal notion of what a ``problem is''. An abstract problem
is a relation $Q$ between a set of instances $I$ and a set of solutions $S$.
The relation is not injective, as eg. an instance of the SHORTEST-PATH
problem might have several solutions.

We only concentrate on decision problems. Where $S = \{0,1\}$. A such problem
for SHORTEST-PATH would take an additional integer $k$ and will return whether
there is a path shorter than $k$ length.

When we have an optimization we can usually make a decision problem by saying
``is there a solution at least as good as $k$'' for some $k$ (not necessarily
an integer).


\subsection{Encodings}
We want our problems on computers, so we want our problem instances encoded as
binary strings. That is, we want an encoding $e : I\to \{0,1\}^*$. A problem
for which $I = \{0,1\}^*$ is called a concrete problem. Note that some
instances might not make sense.

We talk about running time in terms of the size of encodings. Eg. if an
algorithm takes an integer $k$ and runs in $\Theta(k)$. If the encoding
is unary (a string of $1$'s) then the running time is $O(n) = |i|$. If $i$
is encoded in binary the running time is $O(2^n) = 2^{|i|}$. This is running
time as a function of \textit{input length} as opposed to \textit{input size}
which could be eg. amount of vertices. Note that input size and input length
with reasonable encodings are often polynomially related, so we will not talk
more about this.

We call two encodings polynomially related if we can compute one from the
other in polynomial time. For instance we can compute a base 3 integer from
a base 2 integer in polynomial time.

If some $e_1, e_2$ are polynomially related, then
$e_1(Q)\in P \Leftrightarrow e_2(Q)\in P$. We can compute $e_1(i)$ from
$e_2(i)$ in polynomial time $O(n^c)$. Then if $|e_2(i)| = n$ we must have
$|e_1(i)| \le n^c$ as the output can't be bigger than the running time. If
$e_1(Q)$ is polynomially solvable we can solve $e_1(i)$ in $|e_1(i)|^k$, so
we get $O(n^{ck})$.\qed

We assume problem instances are encoding in reasonable fashion (eg. not unary).
We denote a standard encoding with angle brackets. For a graph $G$ the standard
encoding is therefore $\langle G \rangle$.

\subsection{Languages}
We use the alphabet $\Sigma = \{0,1\}$ as well as the empty string
$\varepsilon$. The set of all binary strings is denoted 
\[\Sigma^* = \{\varepsilon, 0, 1, 00, 01, 10, 11, 000, \ldots \}\]
We concentrate on languages $L\subseteq \Sigma^*$. We can do various operations
on these languages:

\begin{itemize}
\item $L_1\cup L_2 = \{x : x\in L_1 \lor x\in L_2\}$
\item $L_1\cap L_2 = \{x : x\in L_1 \land x\in L_2\}$
\item $\overline{L} = \Sigma^* \setminus L$
\item $L_1L_2 = \{x_1x_2 : x_1\in L_1\land x_2\in L_2\}$
\item $L^* = \{\varepsilon\} \cup L_1 \cup L_2 \cup \ldots$
\end{itemize}

If we look at a decision problem $Q$ we can view it as a language:

\[
L = \{x\in \Sigma^* : Q(x) = 1\}
\]

For instance the language for PATH would be $\langle G, u, v, k \rangle$ for
all graphs $G$ where there exists a path from $u\leadsto v$ of at most $k$
length.

An algorithm $A$ accepts a language $L = \{x\in \Sigma^* : A(x) = 1\}$. And
rejects a string if $A(x) = 0$. Note that it might not necessarily reject
all strings $x\notin L$ (it might loop forever). An algorithm decides a
language $L$ if for any $x\in L$ it correctly decides whether $A(x) = 1$. It
must therefor accept or reject any string in $L$. We also talk about
polynomial time accept/decision.

We can define $P$ as $P = \{L\subseteq \{0,1\}^* : \text{ There exists an
algorithm that decides $L$ in polynomial time}$.

It is also works if we replace ``decides'' with ``accepts''. The idea is that
if we have an algorithm $A$ that accepts $x$ in $O(n^k)$, we can construct
an algorithm that runs $A$ for $O(n^k)$ time and if it hasn't returned halts
it and return $0$. Note that we might not be able to construct this algorithm
$A'$ very easily.


\section{Polynomial-time verification}
We want to define the complexity class $NP$ as the class of languages for
which we can verify that a solution is correct in polynomial time.

Imagine that along with an instance $\langle G,u,v,k \rangle$ of PATH, we're
also given a path $P$ from $u\leadsto v$. It is easy to verify if this path
satisfies the constraints of the problem. For each $e\in P$ check that
$e\in G.E$ and check that $\sum_{e = (u,v)\in P} w(u,v) \le k$.

We define such a verification algorithm
$A : \{0,1\}^*\times\{0,1\}^*\to\{0,1\}$ as an algorithm that takes a
problem instance $x$ and a syndicate $y$ and outputs whether $y$ is a solution
to $x$ or not. A verification algorithm $A$ verifies a language:
\[
L = \{x\in \{0,1\}^* : \text{ there exists } y\in \{0,1\}^* \text{ such that }
A(x,y) = 1\}
\]

We say that a language $L$ belongs to $NP$ if there exists a polynomial-time
verification algorithm $A$ and a constant $c$, such that:

\[
L = \{x\in \{0,1\}^* : A(x,y) = 1 \text{ for some $y$ with } |y| = O(|x|^c)\}
\]

It is obvious that $P\subseteq NP$. We also have a complexity class co-NP that
consists of $L : \overline{L}\in NP$. We also have $P\in$ co-NP.


\section{NP-Complete and reductions}
We use reductions to reduce a problem instance to an instance of another
probem. We say that a problem $Q$ can be reduced to another problem $Q'$ if
any instance can be easily rephrased. For example a linear equation
$ax + b = 0$ can be rephrased as a quadratic equating easily by adding $0x^2$:
$0x^2 + ax + b = 0$.

If a language $L_1$ can be reduced to another language $L_2$ in
polynomial-time we write $L_1\le_p L_2$. More formally we have a function
$f : \{0,1\}^* \to \{0,1\}^*$ such that $x\in L_1 \Leftrightarrow f(x)\in L_2$.
Because we only discuss decision problems we can easily produce an answer to
$L_1$ from an answer to $L_2$. It is obvious that if $L_2\in P$ and
$L_1\le_p L_2$ then $L_1\in P$.

We say that a language $L$ is NP-Complete if:

\begin{enumerate}
\item $L\in NP$
\item $\forall L'\in NP : L' \le_p L$
\end{enumerate}

If (2) is satisfied for a language $L$ but not necessarily (1), we say the
language is NP-Hard. It is obvious that if any NP-Complete problem is in $P$
then $P = NP$.

\subsection{Circuit satisfiability}
We will show that this problem is NP-Complete and then use reductions to show
that other problems are.

The decision problem we look at is:
\[
\text{CIRCUIT-SAT } = \{\langle C \rangle : C \text{ is satisfiable boolean
    combinational circuit}\}
\]
We say that the size of such a problem is the amount of AND, NOT and OR gates
as well as the amount of wires.

We have that the language belongs to NP. Simply compute the gates' output
values according to the syndicate provided. This can be done in linear time.
We must also note that $|y| = O(|x|^c)$ which is clear because $|y|$ only
provides the value for some wire, but the amount of wires is strictly less
than $|x|$.

\begin{proof}
$L\in NP$ so we must have a verification algorithm $A$ that runs in polynomial
time.

The idea is now to create a combinational circuit that ``simulates'' $A$.

\begin{enumerate}
\item Let $n = |x|$ and let $T(n)$ be the running time of $A$.
\item Let $c_i$ be the configuration $i$th configuration of $A$. That is
    the program counter, $x$, $y$, the state, etc.
\item Let $M$ be the the circuit that implements the computer hardware maps
    each configuration $c_i$ to the next $c_{i+1}$.
\item If $A$ runs in $T(n)$ steps then its output must be somewhere in
    $c_{T(n)}$.
\item Create a combinational circuit consisting of $T(n)$ instances of $M$.
    Hardwire the input $x$ and leave space for input $y$. Ignore all output
    but the one bit that specifies the answer of $A$.
\end{enumerate}

Suppose that there exists a $y$ with $|y| = O(n^k)$. Then $C(y) = A(x,y) = 1$.
If there is input $y$ such that $C(y) = 1$ then we must also have $A(x,y) = 1$.

For the running time. $A$ has constant size independent of $|x|$. Also the
size needed to represent $c_i$ is polynomial in $n$ (otherwise $A$ wouldn't
be polynomial-time). The 

Assuming that constructing $M$ takes polynomial time we can construct the
entire ciruit in polynomial time because we need to construct $O(n^k)$
copies of $M$.
\end{proof}


\section{NP-Completeness proofs}
If a language $L'\in NPC$ exists so $L' \le_p L$ then $L$ is NP-Hard. This
is easy to show.

In order to prove that a language $L$ is NP-Complete we therefore need only to
show:

\begin{enumerate}
\item $L\in NP$
\item Pick a language $L'$ that is NP-Complete and prove that $L\le_p L'$. Ie:
\item show a function $f$ that maps instances of $L'$ to instances of $L$.
\item Show for $f$ that $x\in L' \Leftrightarrow f(x)\in L$.
\item Show that $f$ runs in polynomial time.
\end{enumerate}

\subsection{SAT}
This problem is much like Circuit-Sat. We use AND, NOT, OR, implication and
bi-implication.

To show that SAT $\in NP$ we simply make an algorithm $A$ that assigns the
values and checks the result.

In order to produce a formula from a circuit we could recursively express the
formula of the final gate by the values of the input wires. This is however
exponential. Instead we create a variable for each wire. If we have an AND-gate
with $x_1, x_2$ as inputs we define variable $x_3$ and create the formula
$(x_3 \Leftrightarrow (x_1\land x_2))$. We do this for each gate and connect
then with $\land$. We lastly add the final output wire to the series of
$\land$.

See example p. 1081. It is easy to see that this formula is satisfiable iff the
circuit is.

\subsection{3-CNF}
3-CNF is a formula which is a series of AND clauses each of which is an OR
clause of exactly three distinct variables.

\begin{proof}
The following three step algorithm changes any formula to a 3-CNF:

\begin{enumerate}
\item Create a binary parse tree of the formula. Note that
    $(x_1\lor x_2\lor x_3)$ is the same as $(x_1\lor (x_2\lor x_3))$. Therefore
    we can create a binary tree.
\item View the tree as combinational circuit and create a formula like before.
    Note that this formula will have at most 3 literals in each clause. This
    finishes step 1.
\item For each clause in the formula from step 1 $\phi_i$ we create a
    formula equal to $\neg\phi_i$. We do this by writing the truth table. It
    will have at most 8 lines so we will get a formula of most $O(1)$ clauses
    for each $\phi_i$. It will also give clauses of the form
    $y_1 \lor y_2 \lor y_3$ where each $y_j$ is an or clause with at most
    three variables.
\item Apply demorgans laws to get 3-CNF. If a clause has only 2 variables just
    add a variable $p$: $(x_1\lor x_2\lor p) \land (x_1\lor x_2\lor \neg p)$.
    If there's only one variable add an extra var $q$.
\end{enumerate}

Because each step preserves satisfiasbility it is clear that SAT
$\Leftrightarrow$ 3-CNF.

We already know that step 1 is done in polynomial time. We saw that step 2 is
at most $8$ times as big, so it is also in polynomial time. The last step can
be done in linear time.
\end{proof}



\section{More proofs zzz}

In this section we show that some problems can be reduced to others, blabla.
I have not given verification algorithms, but these are easy.

\subsection{Clique problem}
Given a graph $G$ find the largest $k$ such that a subgraph $V'\subseteq V$
has edges for all $u,v\in V'$. The related decision problem is given a $k$ to
return if a clique of size $k$ exists.

\begin{proof}
We wish to show 3-CNF-SAT $\le_p$ CLIQUE.

For each $\phi_i$ in the formula $\phi$ we create three vertices in a graph
$G$. We put an edge between two vertices $u, v$ if the following two holds:

\begin{itemize}
\item $u,v$ are in different triples (clauses in the 3-CNF).
\item $u$ is not the negation of $v$ in the 3-CNF.
\end{itemize}

This graph can easily be built on polynomial time.

We want to show that $G$ has a clique of size $k$ (the amount of clauses in
$\phi$) iff $\phi$ is satisfiable: If $\phi$ is satisfiable there is a variable
from each clause such that these variables are consistent. If we pick them as
a clique they must therefore have edges between all of them. If there is a
clique of size $k$ we know that there can be at most one vertex from each
triplet. Therefor we must have a vertex from each triplet and they make up
a solution.
\end{proof}

\subsection{Vertex cover problem}
Given a graph $G= (V,E)$ we wish to find a set $V'\subseteq V$ such that for
any edge $(u,v)\in E$ either $u$ or $v$ is in $V'$ minimizing the size of $V'$.
The decision problem is to check if a $V'$ of size $k$ exists.

\begin{proof}
Let $\overline{G}$ define the complement of $G$, which is
$\overline{E} = \{(u,v) : u,v\in V, u\ne v, (u,v)\notin E\}$ and the same $V$.

The claim is that for an instance $\langle G, k \rangle$ of CLIQUE, the
instance $\langle \overline{G}, |V| - k \rangle$ of VERTEX-COVER is a
reduction. Obviously this can be done in polynomial time.

If $V'$ is a clique of size $k$ in $\ol{G}$ then $V\setminus V'$ is a
vertex-cover of $G$. For any edge $(u,v)\in \ol{E}$ we have that either $u$
or $v$ is in $V\setminus V'$, so all edges in $\ol{E}$ must be covered. If not
we would have $u,v\in V'$ but not $u,v\in E$.

Conversely if $V \setminus V'$ is a vertex cover. If $u,v\in V$ and
$u,v\notin V'$ then we must have $(u,v)\notin \ol{E}$ and therefore
$(u,v)\in E$.
\end{proof}

\subsection{Hamilton Cycle}
The problem is whether a simple path exists that visits each vertex exactly
once. We wish to show VERTEX-COVER $\le_p$ HAM-CYCLE

The idea is to create a graph consiting of $|E$ widgets (see fig. p. 1092).
Each of these widgets has exactly simple three paths through it. For each
vertex we create a path between $(u,u^{(i)},6), (u,u^{(i+1)}, 1)$ where
$u^{(i)}$ is the ith vertex connected to $u$. This way, if we select
$u$ to be in the vertex cover we can make a path through all of widgets
corresponding to edges $u$ are connected to. If $u$ covers all edges we can
thus make a hamilton path through all the vertices by starting in
$(u,u^{(1)}, 1)$. We also add $k$ selector vertices and connect each one to
$(u,u^{(1)}, 1)$ and $(u,u^{(\text{degree}(u))}, 6)$.

Proof not added as it is not pensum.

\subsection{TSP}
Decision problem is a tour of length at most $k$.

Given any instance of hamilton cycle we create the graph
$c(i,j) = 0, (i,j)\in E$ and $c(i,j) = 1$ otherwise. If there is a TSP tour of
length $0$ there is a hamilton cycle.

\subsection{Subset-Sum}
Given a set of integer $S$ and an integer $t>0$. Is there a subset
$S'\subseteq S$ which elements sum to $t$?

We want to show 3-CNF-SAT $\le_p$ SUBSET-SUM. Assume no clause contains both
$x_i$ and $\neg x_i$. Also each $x_i$ is in at least one clause.

\begin{proof}
Create an instance of SUBSET-SUM as follows:

\begin{enumerate}
\item For each $x_i\in \phi$ create two base 10 numbers with $n+k$ digits like
    so:
\begin{itemize}
\item Set the $n$ most significant digits to $0$ except for digit $i$. Set this
    to one.
\item Let each of the $k$ least significant digits correspond to a clause in
    $\phi$. If $x_i\in C_k$ then set the least significant digit to $1$, if
    $x_i\in C_{k-1}$ set the second least significant digit to $1$, etc.\\
    For the other variable if $\neg x_i\in C_k$ set the least significant digit
    to $1$, etc.\\
    Set all other digits to $0$.
\item Call these vars $v_i$ and $v_i'$
\end{itemize}
\item For each clause $C_j$ create two variables with all digits to zero except
    the $k-j$th least significant digit. One variable has a $1$ here, one has
    a $2$. Let these vars be $s_1$ and $s_1'$ (s for slack).
\end{enumerate}

\textit{These are unique, which is very important because $S$ is a set!}

Not that if we sum all these numbers no digit will be greater than $6$ (3
variables per clause plus the $3$ from the two slack variables). We can
therefore just use any base strictly greater than $6$ and we won't get carries
when adding.

The claim is now, that if we set $t$ to be $n$ $1$'s followed by $k$ $4$'s
then there will be $S'\subseteq S$ that sums to $t$ iff $\phi$ is satisfiable.

Assume $\phi$ has a satisfying assignment. If $x_i = 1$ in this, include $v_i$
in $S'$ otherwise include $v_i'$. If we now sum up $S'$ we will have $1$'s in
the first $n$ digits (obviously). Because all clauses are satisfied there must
be at least $1$ in the last $k$ digits as well. We can now just add slack
variables untill we get $4$ in each of those digits.

Suppose that there is $S'\subseteq S$ that sums to $t$. There must be exactly
one of $v_i$ or $v_i'$ in $S'$ for each $i$. Also because the slack variables
sum to $3$ each clause is satisfied.
\end{proof}

\end{document}
